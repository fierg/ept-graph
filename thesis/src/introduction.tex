\chapter{Introduction}
\label{ch:Introduction}
%% ==============================
%% ==============================
\section{Motivation}
%% ==============================
\label{ch:Introduction:sec:Motivation}
Edge periodic temporal graphs are graphs where each edge is associated with a binary string that indicates in which time-step the edge is present.
They allow us to identify and capture recurring patterns in temporal data.
By detecting periodicity, we can gain insights into regular behaviors or events that occur at fixed intervals.
This can be valuable in fields such as finance, transportation, and social networks, where understanding periodic trends can aid in decision-making and prediction.
Temporal graphs can become quite complex as they evolve over time and labels for each edge can have different lengths, potentially resulting in the entire sequence repeating only after an exponential period.
By transforming them into edge periodic temporal graphs, we can simplify the analysis of e.g. finding regularities by focusing on the repeating patterns instead of dealing with the entire temporal range.
This reduction in complexity can lead to more efficient algorithms and computational benefits.
In scenarios where resources need to be allocated or scheduled based on temporal patterns, edge periodic temporal graphs enable more efficient usage.
For example, optimizing transportation routes based on recurring traffic patterns can significantly reduce congestion and improve resource allocation.
Furthermore from a data compression standpoint, the conversion leads to more compact representations.
Rather than storing the entire temporal history, edge periodic temporal graphs capture the essential periodic patterns, reducing storage and memory requirements while preserving critical information.
In conclusion, the transformation of temporal graphs into edge periodic temporal graphs serves as a powerful technique that unlocks a variety of applications.

Imagine standing at a train station connecting place $A$ to place $B$, departing every half and full hour.
The schedule gives information about a leaving train for any time-step $t$ for intervals of 5 minutes.
To cover all possible time-steps in a day, a long index with all steps is listed, where you have to search for your exact arrival time, to see when the next one is leaving as this is not apparent in the large table with many additional time points.
Now imagine a different schedule, showing and periodically repeating after one hour.
To find the next train, one just has to check at which time of an hour a train departs, which is not only more concise but also easier to understand for human interpreters.
%% ==============================
\section{Problem Statement}
%% ==============================
\label{ch:Introduction:sec:Problem statement}
To use the benefits of the periodic representation as an edge periodic graph, a transformation from temporal to edge periodic has to be performed.
This transformation can be done in different ways, but we focus on a method based on the decomposition of a deterministic finite automaton.
Reading the labels of a temporal graph and interpreting it as \DFA enables the application of known methods for \DFA decomposition.
Labels from a temporal graph, consisting of binary strings, can be interpreted as unary \DFAs by representing the 1 values as final states of a \DFA. 
\begin{figure}[h]
	\begin{minipage}[t]{0.49\textwidth}
		\centering
		\begin{tikzpicture}[every edge quotes/.style = {auto, font=\footnotesize, sloped}]
			\begin{scope}[every node/.style={circle,fill}]
				\node (A) at (0,0) {};
				\node (B) at (0,2) {};
				\node (C) at (2,0) {};
				\node (D) at (2,2) {};
				\node (E) at (1,3) {};
			\end{scope}
			
			\draw (A) edge["101\dots"] (B)
			(B) edge["100\dots"] (C)
			(D) edge["001\dots"] (C)
			(A) edge["000\dots"] (C)
			(D) edge["111\dots"] (E)
			(B) edge["011\dots"] (D)
			(B) edge["101\dots"] (E);	
		\end{tikzpicture}
		\\ Temporal graph with long edge labels
	\end{minipage}
	\begin{minipage}[t]{0.49\textwidth}
		\centering
		\begin{tikzpicture}[every edge quotes/.style = {auto, font=\footnotesize, sloped}]
			\begin{scope}[every node/.style={circle,fill}]
				\node (A) at (0,0) {};
				\node (B) at (0,2) {};
				\node (C) at (2,0) {};
				\node (D) at (2,2) {};
				\node (E) at (1,3) {};
			\end{scope}
			
			\draw (A) edge["101"] (B)
			(B) edge["100"] (C)
			(D) edge["0010"] (C)
			(A) edge["0"] (C)
			(D) edge["110"] (E)
			(B) edge["01111"] (D)
			(B) edge["10"] (E);
		\end{tikzpicture}
	\\ EPG with short and periodic labels
	\end{minipage}
\end{figure}

Then the \DFA decomposition algorithms can be applied, to get a family of smaller \DFAs, which combined resembles the language of the original \DFA.
These so-called factors each represent a periodic part of the input and can be used to get a shorter and periodic label.
Since we obtain a decomposition family, we might need to replace the original temporal graph with an edge periodic graph with multi-edges. 

%% ==============================
\section{Structure of this Work}
%% ==============================
\label{ch:Intoduction:sec:Structure}
This work is structured into a first introduction to notation and a survey of the current state of the art of temporal and edge periodic graph analysis as well as the decomposition of \DFAs.
Following this, the used real-world data sets are described, in particular the F2F graphs.
The core of the document revolves around the decomposition of edge labels by decomposing \DFAs.
Section \ref{ch:analysis} explains the decomposition of \DFAs and unary \DFAs, while Section \ref{ch:analysis:max-divisors} explores a decomposition method employing maximal divisors.
A dedicated section (Section \ref{chap:and-vs-or}) compares AND vs OR decomposition.
The aspect of \enquote{explainability} and explainable edge labels or decompositions for human analysis is thoroughly examined in Section \ref{ch:explainability}.
This includes the introduction of an explainability metric and the methodology for measuring explainability.
Novel approaches to the problem are presented in Section \ref{ch:novel-algos}, including Greedy Short Factors and Fourier Transform 
Additionally, a problem generalization is formally analyzed.
The subsequent sections detail the implementation of these approaches in Section \ref{ch:Implementation} and their evaluation in Section \ref{ch:Evaluation}.
The evaluation includes both performance and decomposition assessments, with a breakdown of different methods of decomposition.
Furthermore, the document assesses the explainability of the proposed methods.
Finally, the document concludes with a discussion (Section \ref{ch:Discussion}) where the findings are contextualized, and implications are explored.






