\chapter{Introduction}
\label{ch:Introduction}
%% ==============================
%% ==============================
\section{Motivation}
%% ==============================
\label{ch:Introduction:sec:Motivation}
Edge periodic temporal graphs allow us to identify and capture recurring patterns in temporal data. By detecting periodicity, we can gain insights into regular behaviors or events that occur at fixed intervals. This can be valuable in fields such as finance, transportation, and social networks, where understanding periodic trends can aid in decision-making and prediction. Also the analysis is easier for EPG's. Temporal graphs can become quite complex as they evolve over time. By transforming them into edge periodic temporal graphs, we can simplify the analysis by focusing on the repeating patterns instead of dealing with the entire temporal range. This reduction in complexity can lead to more efficient algorithms and computational benefits. Resource management also greatly benefits from this transformation. In scenarios where resources need to be allocated or scheduled based on temporal patterns, edge periodic temporal graphs enable more efficient usage. For example, optimizing transportation routes based on recurring traffic patterns can significantly reduce congestion and improve resource allocation. Furthermore from a data compression standpoint, the conversion leads to more compact representations. Rather than storing the entire temporal history, edge periodic temporal graphs capture the essential periodic patterns, reducing storage and memory requirements while preserving critical information. In conclusion, the transformation of temporal graphs into edge periodic temporal graphs serves as a powerful technique that unlocks a variety of applications. Additionally there is a growing interest in EPGs due to their potential applications in various fields such as biology, physics, and computer science. For instance, EPGs can be used to study the periodic behavior of biological systems such as gene regulatory networks, where genes can be turned on and off in a cyclic manner. In physics, EPGs can model the periodic behavior of particles in a magnetic field, where particles move in a repeating pattern over time. Furthermore, the study of EPGs can lead to the development of more efficient algorithms for analyzing and processing dynamic networks. EPGs can also provide a deeper understanding of the behavior of dynamic networks, which can help in the design of more effective strategies for controlling and optimizing such networks. Overall, the field of EPGs presents a rich area for research with numerous potential applications and benefits.
%% ==============================
\section{Problem statement}
%% ==============================
\label{ch:Introduction:sec:Problem statement}

\begin{figure}[h]
	\begin{minipage}[t]{0.49\textwidth}
		\centering
		\begin{tikzpicture}[every edge quotes/.style = {auto, font=\footnotesize, sloped}]
			\begin{scope}[every node/.style={circle,fill}]
				\node (A) at (0,0) {};
				\node (B) at (0,2) {};
				\node (C) at (2,0) {};
				\node (D) at (2,2) {};
				\node (E) at (1,3) {};
			\end{scope}
			
			\draw (A) edge["101\dots"] (B)
			(B) edge["100\dots"] (C)
			(D) edge["001\dots"] (C)
			(A) edge["000\dots"] (C)
			(D) edge["111\dots"] (E)
			(B) edge["011\dots"] (D)
			(B) edge["101\dots"] (E);	
		\end{tikzpicture}
		\\ Temporal graph with long edge labels
	\end{minipage}
	\begin{minipage}[t]{0.49\textwidth}
		\centering
		\begin{tikzpicture}[every edge quotes/.style = {auto, font=\footnotesize, sloped}]
			\begin{scope}[every node/.style={circle,fill}]
				\node (A) at (0,0) {};
				\node (B) at (0,2) {};
				\node (C) at (2,0) {};
				\node (D) at (2,2) {};
				\node (E) at (1,3) {};
			\end{scope}
			
			\draw (A) edge["101"] (B)
			(B) edge["100"] (C)
			(D) edge["0010"] (C)
			(A) edge["0"] (C)
			(D) edge["110"] (E)
			(B) edge["01111"] (D)
			(B) edge["10"] (E);
		\end{tikzpicture}
	\\ EPG with short and periodic labels
	\end{minipage}

\end{figure}



To use the benefits of EPG's, a transformation from temporal to egde periodic has to be performed. This decomposition can be split up into multiple steps. Initially, we read the label and interpret it as a unary automata. Labels consisting of binary values can interpreted as unary DFAs by representing either the 0 or 1 values as final states of the DFA, which representation will result in better EPG transformations will be a topic later on. TODO 

%% ==============================
\section{Structure of this Work}
%% ==============================
\label{ch:Intoduction:sec:Structure}
This work is structured into a first introduction into the basics of EPGs and an analysis of the current state of the art. Then, the conceptual design is depicted with a following analysis of the results. Afterwards, the implementation of the algorithms are described and the work as a whole is evaluated with a short closing discussion.  

TODO

