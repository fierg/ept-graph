\chapter{Introduction}
\label{ch:Introduction}
%% ==============================
%% ==============================
\section{Motivation}
%% ==============================
\label{ch:Introduction:sec:Motivation}
There is a growing interest in EPGs due to their potential applications in various fields such as biology, physics, and computer science. For instance, EPGs can be used to study the periodic behavior of biological systems such as gene regulatory networks, where genes can be turned on and off in a cyclic manner. In physics, EPGs can model the periodic behavior of particles in a magnetic field, where particles move in a repeating pattern over time. Furthermore, the study of EPGs can lead to the development of more efficient algorithms for analyzing and processing dynamic networks. EPGs can also provide a deeper understanding of the behavior of dynamic networks, which can help in the design of more effective strategies for controlling and optimizing such networks. Overall, the field of EPGs presents a rich area for research with numerous potential applications and benefits.
%% ==============================
\section{Problem statement}
%% ==============================
\label{ch:Introduction:sec:Problem statement}

\begin{minipage}[t]{0.49\textwidth}
\begin{tikzpicture}[every edge quotes/.style = {auto, font=\footnotesize, sloped}]
	\begin{scope}[every node/.style={circle,fill}]
		\node (A) at (0,0) {};
		\node (B) at (0,2) {};
		\node (C) at (2,0) {};
		\node (D) at (2,2) {};
		\node (E) at (1,3) {};
	\end{scope}
	
	\draw (A) edge["101\dots"] (B)
	(B) edge["100\dots"] (C)
	(D) edge["001\dots"] (C)
	(A) edge["000\dots"] (C)
	(D) edge["111\dots"] (E)
	(B) edge["011\dots"] (D)
	(B) edge["101\dots"] (E);	
\end{tikzpicture}
\end{minipage}
\begin{minipage}[t]{0.49\textwidth}
\begin{tikzpicture}[every edge quotes/.style = {auto, font=\footnotesize, sloped}]
	\begin{scope}[every node/.style={circle,fill}]
		\node (A) at (0,0) {};
		\node (B) at (0,2) {};
		\node (C) at (2,0) {};
		\node (D) at (2,2) {};
		\node (E) at (1,3) {};
	\end{scope}
	
	\draw (A) edge["101"] (B)
	(B) edge["100"] (C)
	(D) edge["0010"] (C)
	(A) edge["0"] (C)
	(D) edge["110"] (E)
	(B) edge["01111"] (D)
	(B) edge["10"] (E);
\end{tikzpicture}
\end{minipage}



%% ==============================
\section{Structure of this Work}
%% ==============================
\label{ch:Intoduction:sec:Structure}
\par{
This work is structured into a first introduction into the basics of EPGs and an analysis of the current state of the art. Then, the conceptual design is depicted with a following analysis of the results. Afterwards, the implementation of the algorithms are described and the work as a whole is evaluated with a short closing discussion.  
}
