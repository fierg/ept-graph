\chapter{Decompositions of Edge Labels by decomposing Unary DFAs}
\label{ch:Analysis}
%% ==============================

As our goal is to represent general dynamic networks and temporal graphs as EPGs, one problem is the missing periodicity in general temporal and dynamic graphs. 

\begin{figure}[h]
	\begin{minipage}[t]{0.49\textwidth}
		\centering
		\begin{tikzpicture}[every edge quotes/.style = {auto, font=\footnotesize, sloped}]
			\begin{scope}[every node/.style={circle,fill}]
				\node (A) at (0,0) {};
				\node (B) at (0,2) {};
				\node (C) at (2,0) {};
				\node (D) at (2,2) {};
				\node (E) at (1,3) {};
			\end{scope}
			
			\draw (A) edge["101\dots"] (B)
			(B) edge["100\dots"] (C)
			(D) edge["001\dots"] (C)
			(A) edge["000\dots"] (C)
			(D) edge["111\dots"] (E)
			(B) edge["011\dots"] (D)
			(B) edge["101\dots"] (E);	
		\end{tikzpicture}
	\end{minipage}
	\begin{minipage}[t]{0.49\textwidth}
		\centering
		\begin{tikzpicture}[every edge quotes/.style = {auto, font=\footnotesize, sloped}]
			\begin{scope}[every node/.style={circle,fill}]
				\node (A) at (0,0) {};
				\node (B) at (0,2) {};
				\node (C) at (2,0) {};
				\node (D) at (2,2) {};
				\node (E) at (1,3) {};
			\end{scope}
			
			\draw (A) edge["101"] (B)
			(B) edge["100"] (C)
			(D) edge["0010"] (C)
			(A) edge["0"] (C)
			(D) edge["110"] (E)
			(B) edge["01111"] (D)
			(B) edge["10"] (E);
		\end{tikzpicture}
	\end{minipage}
	\caption{TODO}
\end{figure}

To transform a temporal graph, a shortest possible string $w'$ which is equal to the original label at every time step $\forall t \geq 0, \tau(e)[t] = w'[t]$ has to be found. This $w'$ can itself be composed by combining different factors $w_1,w_2,\dots,w_n$. Initially, we start with a temporal graph where all edge labels $\tau(e)$ have fixed length and there are no periods present. To find and analyze such periods in the given labels, the algorithm from \cite{DBLP:journals/corr/abs-2107-04683} is used and adapted to our problem. To apply the algorithm which is defined for automata, a label $ w \in \{0,1\}^*$ is interpreted as an unary automata. In the label either the $0s$ or the $1s$ symbols are used to represent final states $Q_f$. The algorithm from \cite{DBLP:journals/corr/abs-2107-04683} can be simplified due to the fact that we only represent unary automata and therefore only have a single transition from each state, basically forming a simple circle of all possible states. Using the fact that our alphabet is of size one, $|\Sigma| = 1$ we only need to follow a single transition and furthermore, we do not need to check each state separately, we only need to check multiples of the chosen period. This means that for a period length of $i$ we only have to check $i$ states on the circle being in the same state.

\section{Decompositions of Unary DFAs}

A DFA $\mathcal{A}$ is composite if its language $L(\mathcal{A})$ can be decomposed into an intersection $\cup^k_{i=1} L(\mathcal{A}_i)$ of languages of smaller DFAs. Otherwise, $\mathcal{A}$ is prime. This notion of primality was introduced by Kupferman and Mosheiff in \cite{prime-languages}, and they proved that we can decide whether a DFA is composite in ExpSpace and later in \cite{unara-prime-languages}, the decomposition question for unary DFAs was proven to be in LOGSPACE. In the work \cite{DBLP:journals/corr/abs-2107-04683} by Jecker, Mazzocchi and Wolf, where they provided a LOGSPACE algorithm for commutative permutation DFAs, if the alphabet size is fixed which also puts the bounded $k$-composite question in the LOGSPACE complexity class, see \ref{tab:dfa-decomp-complexity} for reference.

\subsection{Decomposing using maximal divisors}
Following the LOGSPACE-algorithm solving the Bound-Decomp problem for unary DFAs from \cite{DBLP:journals/corr/abs-2107-04683}


\section{Combining factors into a EPG label}

Now with the collection of periods, novel problems arises. How to combine the periods in the most optimal manner? What is the minimal number of periods needed to cover a given array? These and other questions where tried to answer both theoretically and empirically with some real world data.

