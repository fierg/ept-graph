\chapter*{Abstract}
%% ==============================
Edge periodic temporal graphs (EPGs) are graphs with varying edges over time, that can represent the periodic behavior of dynamic networks. In EPGs, for each edge $e$ of the graph, a binary string $s_e$ determines in which time steps the edge is present, namely, $e$ is present in time step $t$ if and only if $s_e$ contains a 1 at position $t~ mod~ |s_e|$. Due to their periodic nature, EPGs can be a more concise way of representing complex periodic systems compared to traditional temporal graphs, as they include the entire sequence of graphs. In this paper, we study the effectiveness of representing dynamic networks as EPGs theoretically and empirically.

\paragraph{Übersetzung}
Edge-periodische Zeitgraphen (EPGs) sind Graphen mit sich über die Zeit ändernden Kanten, die das periodische Verhalten von dynamischen Netzwerken repräsentieren können. In EPGs bestimmt ein binäres Wort $s_e$ für jede Kante $e$ des Graphen, in welchen Zeitpunkten die Kante vorhanden ist. Konkret ist die Kante $e$ im Zeitpunkt $t$ vorhanden, wenn $s_e$ an der Position $t~ mod~ |s_e|$ eine 1 enthält. Aufgrund ihrer periodischen Natur können EPGs eine prägnantere Möglichkeit zur Darstellung komplexer periodischer Systeme im Vergleich zu herkömmlichen zeitlichen Graphen darstellen, da sie die gesamte Sequenz der Graphen enthalten. In dieser Arbeit untersuchen wir theoretisch und empirisch die Effektivität der Darstellung dynamischer Netzwerke als EPGs.