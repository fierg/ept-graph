\chapter*{Abstract}
%% ==============================
Edge periodic temporal graphs (EPGs) are graphs edge availability varies over time, that can represent the periodic behavior of dynamic networks. In EPGs, for each edge $e$ of the graph, a binary string $s_e$ determines in which time steps the edge is present, namely, $e$ is present in time step $t$ if and only if $s_e$ contains a 1 at position $t~ mod~ |s_e|$. Due to their periodic nature, EPGs can be a more concise way of representing complex periodic systems compared to traditional temporal graphs. In this paper, we study the effectiveness of representing dynamic networks as EPGs theoretically and empirically, by interpreting the labels as unary automata and applying a decomposition algorithm for DFAs as well as well as two novel algorithms that identify other, more \enquote{explainable} factors for human interpreters.

\paragraph{Übersetzung}
Edge-periodische Zeitgraphen (EPGs)  sind Graphen, deren Kantenverfügbarkeit sich im Laufe der Zeit ändert und die das periodische Verhalten dynamischer Netzwerke darstellen können. In EPGs bestimmt für jede Kante $e$ des Graphen eine binäre Zeichenkette $s_e$, in welchen Zeitschritten die Kante vorhanden ist; das heißt, $e$ ist im Zeitschritt $t$ vorhanden, wenn und nur wenn $s_e$ an Position $t~ mod~ |s_e|$ eine 1 enthält. Aufgrund ihrer periodischen Natur können EPGs eine prägnantere Möglichkeit darstellen, komplexe periodische Systeme im Vergleich zu herkömmlichen zeitlichen Graphen zu repräsentieren. In dieser Arbeit untersuchen wir theoretisch und empirisch die Effektivität der Darstellung dynamischer Netzwerke als EPGs, indem wir die Labels als unäre Automaten interpretieren und einen Dekompositionsalgorithmus für DFAs anwenden, sowie zwei neuartige Algorithmen, die andere, für menschliche Interpreten potenziell \enquote{erklärbarere} Faktoren identifizieren.