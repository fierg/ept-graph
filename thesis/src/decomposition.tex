\chapter{Decomposition of Edge Labels}
\label{ch:Analysis}
%% ==============================

As our goal is to represent general dynamic networks and temporal graphs as EPGs, one problem is the missing periodicity in general temporal and dynamic graphs. 

\begin{figure}[h]
	\begin{minipage}[t]{0.49\textwidth}
		\centering
		\begin{tikzpicture}[every edge quotes/.style = {auto, font=\footnotesize, sloped}]
			\begin{scope}[every node/.style={circle,fill}]
				\node (A) at (0,0) {};
				\node (B) at (0,2) {};
				\node (C) at (2,0) {};
				\node (D) at (2,2) {};
				\node (E) at (1,3) {};
			\end{scope}
			
			\draw (A) edge["101\dots"] (B)
			(B) edge["100\dots"] (C)
			(D) edge["001\dots"] (C)
			(A) edge["000\dots"] (C)
			(D) edge["111\dots"] (E)
			(B) edge["011\dots"] (D)
			(B) edge["101\dots"] (E);
		\end{tikzpicture}\\
		Temporal graph with long labels
	\end{minipage}
	\begin{minipage}[t]{0.49\textwidth}
		\centering
		\begin{tikzpicture}[every edge quotes/.style = {auto, font=\footnotesize, sloped}]
			\begin{scope}[every node/.style={circle,fill}]
				\node (A) at (0,0) {};
				\node (B) at (0,2) {};
				\node (C) at (2,0) {};
				\node (D) at (2,2) {};
				\node (E) at (1,3) {};
			\end{scope}
			
			\draw (A) edge[bend left, below,"101"] (B)
			(A) edge[bend right, below,"0001"] (B)
			(B) edge["100"] (C)
			(D) edge[bend right, below,"0100"] (C)
			(D) edge[bend left, below,"10"] (C)
			(A) edge["0"] (C)
			(D) edge[bend right, above,"110"] (E)
			(B) edge[bend right, below,"01111"] (D)
			(B) edge["10"] (E)
			(E) edge[bend right, below,"01"]  (D);
		\end{tikzpicture}\\
		EPG with multi edges but short labels
	\end{minipage}
	\caption{Desired transformation of temporal graphs into EPGs}
\end{figure}

To transform a temporal graph, a shortest possible string $w'$ which is equal to the original label at every time step $\forall t \geq 0, \tau(e)[t] = w'[t]$ has to be found. This $w'$ can itself be composed by combining different factors $w_1,w_2,\dots,w_n$. Initially, we start with a temporal graph where all edge labels $\tau(e)$ have fixed length and there are no periods present. To find and analyze such periods in the given labels, the algorithm from \cite{DBLP:journals/corr/abs-2107-04683} is used and adapted to our problem. To apply the algorithm which is defined for automata, a label $ w \in \{0,1\}^*$ is interpreted as an unary automata. In the label either the $0s$ or the $1s$ symbols are used to represent final states $Q_f$. The algorithm from \cite{DBLP:journals/corr/abs-2107-04683} can be simplified due to the fact that we only represent unary automata with \textbf{$|\Sigma| = 1$} and therefore only have a single transition from each state, basically forming a simple circle of all possible states. Using the fact that our alphabet is of size one, we only need to follow a single transition and furthermore, we only need to check multiples of the chosen period. This means that for a period length of $i$ we only have to check $i$ states on the circle being in the same state.

\section{Decomposition of \DFAs and Unary \DFAs}
\label{sec:decomposition-unary-dfas}

\begin{figure}[h]
	\begin{minipage}[t]{0.29\textwidth}
		\centering
		Label $w=1001$
	\end{minipage}
	\begin{minipage}[t]{0.69\textwidth}
		\centering
		\begin{tikzpicture}[shorten >=1pt,node distance=2cm,on grid,auto] 
			\node[state,initial,accepting] (q_0)   {$q_0$}; 
			\node[state] (q_1) [right=of q_0] {$q_1$}; 
			\node[state] (q_2) [right=of q_1] {$q_2$}; 
			\node[state,accepting](q_3) [right=of q_2] {$q_3$};
			\path[->] 
			(q_0) edge  node {} (q_1)
			(q_1) edge  node {} (q_2)
			(q_2) edge  node {} (q_3)
			(q_3) edge[bend right, above]  node {} (q_0);
		\end{tikzpicture}
		\begin{tikzpicture}[shorten >=1pt,node distance=2cm,on grid,auto] 
			\node[state,initial] (q_0)   {$q_0$}; 
			\node[state,accepting] (q_1) [right=of q_0] {$q_1$}; 
			\node[state,accepting] (q_2) [right=of q_1] {$q_2$}; 
			\node[state](q_3) [right=of q_2] {$q_3$};
			\path[->] 
			(q_0) edge  node {} (q_1)
			(q_1) edge  node {} (q_2)
			(q_2) edge  node {} (q_3)
			(q_3) edge[bend right, above]  node {} (q_0);
		\end{tikzpicture}
	\end{minipage}
	\caption{Equivalence of binary strings and unary permutation \DFAs}
\end{figure}

A DFA $\mathcal{A}$ is composite if its language $L(\mathcal{A})$ can be decomposed into an intersection $\cup^k_{i=1} L(\mathcal{A}_i)$ of languages of smaller DFAs. Otherwise, $\mathcal{A}$ is prime. This notion of \textit{primality} was introduced by Kupferman and Mosheiff in \cite{prime-languages}, and they proved that we can decide whether a DFA is composite in \ExpSpace and later in \cite{unara-prime-languages}, the decomposition question for unary DFAs was proven to be in \LogSpace. In the paper~\cite{DBLP:journals/corr/abs-2107-04683} by Jecker, Mazzocchi and Wolf, they provided a \LogSpace algorithm for commutative permutation DFAs, if the alphabet size is fixed, which also puts the bounded $k$-composite question for unary DFAs in the \LogSpace complexity class, see Table \ref{tab:dfa-decomp-complexity} for further reference. The decomposition of unary \DFAs are usually characterized by means of clean quotients. Let  $A = \lbrace\Sigma, Q, q_I , \sigma, F\rbrace$ be a unary-DFA. A clean quotient $A_d$ of $A$ is a DFA obtained by folding its cycle of length $l$ to a cycle of length $d$, for some strict divisor $d$ of $l$. Formally, $A_d$ is induced by the equivalence relation $\sim_d$ defined by

\[ q_i \sim_d q_j ~\text{if and only if}~i \equiv j ~mod~ d \]

Since $\sim_d$ is coherent with $\sigma$, and therefore $L(A) \subseteq L(A_d)$.

\begin{figure}[h]
	\begin{minipage}[t]{\textwidth}
		\centering
		\begin{tikzpicture}[shorten >=1pt,node distance=2cm,on grid,auto] 
			\node[state,initial] (q_0)   {$q_0$}; 
			\node[state,accepting] (q_1) [ right=of q_0] {$q_1$}; 
			\node[state] (q_2) [ right=of q_1] {$q_2$}; 
			\node[state,accepting] (q_3) [ right=of q_2] {$q_3$};
			\node[state](q_4) [ right=of q_3] {$q_4$};
			\node[state](q_5) [ right=of q_4] {$q_5$};
			\path[->] 
			(q_0) edge  node {} (q_1)
			(q_1) edge  node {} (q_2)
			(q_2) edge  node {} (q_3)
			(q_3) edge  node {} (q_4)
			(q_4) edge  node {} (q_5)
			(q_5) edge[bend right, above]  node {} (q_0);
		\end{tikzpicture}		
	\end{minipage}
	\begin{minipage}[b]{0.39\textwidth}
		\centering
		\begin{tikzpicture}[shorten >=1pt,node distance=2cm,on grid,auto] 
			\node[state,initial] (q_0)   {$q_0$}; 
			\node[state,accepting] (q_1) [right=of q_0] {$q_1$}; 
			\path[->] 
			(q_0) edge  node {} (q_1)
			(q_1) edge[bend right, above]  node {} (q_0);
		\end{tikzpicture}
	\end{minipage}
	\begin{minipage}[b]{0.59\textwidth}
		\centering
		\begin{tikzpicture}[shorten >=1pt,node distance=2cm,on grid,auto] 
			\node[state,initial,accepting] (q_0)   {$q_0$}; 
			\node[state,accepting] (q_1) [right=of q_0] {$q_1$}; 
			\node[state](q_2) [right=of q_1] {$q_2$};
			\path[->] 
			(q_0) edge  node {} (q_1)
			(q_1) edge  node {} (q_2)
			(q_2) edge[bend right, above]  node {} (q_0);
			
		\end{tikzpicture}
	\end{minipage}
	\caption{The DFA $A$ and its clean quotients $A_2$ and $A_3$}
	\label{fig:clean-quotients-example}
\end{figure}

In the example in figure \ref{fig:clean-quotients-example}, the \DFA $A$ with final states $q_1$ and $q_3$ can be replaced by the clean quotients $A_2$ and $A_3$ as there is no $q_i \in F$ in $A$ where both clean quotients $A_d$ are not in a final state $q_{i ~mod~ d}$. Note that the coherency of $\sim$ with respect to $\sigma$ guarantees that the definition of $\sigma'$ is independent of the choice of the state $p$ in $[p]$. On the other hand, we do not require states related by $\sim$ to agree on membership in $F$, and define $F_d$ so that the language of $A_d$ over-approximates that of $A$. Formally, $L(A) \subseteq L(A_d)$, as every accepting run of $A$ induces an accepting run of $A_d$. In this example this is visible and there is a state $q_0$ in $A$ where the clean quotient $A_2$ is in a final state but $A_3$ is not and only the combination of both clean quotients is able to fully decompose the original \DFA. Finding clean quotients is trivial but the number of potential quotients rises linear with the number of states or the length of the edge label. Given a positive integer $n$, let $d(n)$ denote the number of divisors of $n$ including 1 and $n$, so for example $d(8)=4$. More generally, if $n$ has a prime factorization $n = p_1^{a_1} \ldots p_k^{a_k}$ then by the fundamental theorem of arithmetic $d(n) = (a_1+1) \cdot \ldots \cdot (a_k+1)$. Clearly, $d(n) \leq n$ and therefore $d(n) = n^{o(1)}$. As already mentioned, this can be improved by considering only maximal divisors, which will be described in the next chapter.

\section{Decomposition of \DFAs Using Only Maximal Divisors}
The idea of folding \DFAs from \cite{prime-languages} was greatly improved by \cite{DBLP:journals/corr/abs-2107-04683} by limiting the the potential factors to maximal divisors instead of strict divisors of $d$.

TODO: elaborate

\begin{algorithm}[H]
	\label{algo:original-composite}
	\DontPrintSemicolon
	\SetKwProg{Fn}{Function}{:}{}
	\Fn{isBoundedComposite($A = ⟨{a}, Q, qI , \sigma, F ⟩ $: unary DFA, integer k)}{
		\ForEach{binaryString $\in \{0,1\}^{log|Q|}$ with $\leq k$ ones}{
			\If{testCombination($A$,binaryString)}{\KwRet true}
			}
			\KwRet false
		}
		
		\Fn{testCombination($A = ⟨{a}, Q, qI , \sigma, F ⟩ $: unary DFA, binaryString)}{
			\ForEach{$q \in Q \setminus F$}{
				\If{not cover(A,q,binaryString)}{\KwRet false}
				\KwRet true
			}
		}
		
		\Fn{cover($A = ⟨{a}, Q, qI , \sigma, F ⟩ $: unary DFA, binaryString, $q \in Q \setminus F$)}{
			\ForEach{$i$ with wordCombination[i]=1}{
				$p_i \gets i$'th prime divisor of $|Q|$\;
				\If{cover($A,q,\sigma(q,a^{|Q|/p_i})$)}{\KwRet true}
			}
			\KwRet false
		}
		
		\caption{Algorithm solving the Decomp problem for unary DFAs}
\end{algorithm}

Given 2 divisors $i^1$ and $i^2$ of $|Q|$, with $i^1 < i^2 <|Q|$ and $i^1$ divides $i^2$, then all states covered by $a^{i1}$ are also covered by $a^{i2}$. Therefore we now only consider words of the form $a^i$ where $i$ is a maximal divisor of $|Q|$ as potential candidates for the decomposition. Now, let $p_1^{j1} \cdot p_2^{j2} \cdot ~ \dots ~\cdot p_m^{jm} = |Q|$ be the prime factor decomposition of $|Q|$. Since $|Q|$ is given in unary we can compute the prime factor decomposition of $|Q|$ in space logarithmic in $|Q|$. $A$ is $k$-factor composite if and only if a selection of $k$ words from the set $W = \{a^{|Q|/p_i} | 1 \leq i \leq m\}$ cover all the rejecting states of $A$.  As $|W| = m$ is logarithmic in $|Q|$, we can iterate over all sets in $2^W$ of size at most $k$ in \LogSpace. By Lemma 6 from \cite{DBLP:journals/corr/abs-2107-04683}, checking whether a state $q \in Q$ is covered by the current collection of $k$ words can also be done in \LogSpace.

%%% Local Variables: 
%%% mode: latex
%%% TeX-master: "thesis"
%%% End: 