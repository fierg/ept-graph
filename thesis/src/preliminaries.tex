\chapter{Preliminaries}
\label{ch:Preliminaries}
%% ==============================

\section{Notation}
For a string $w = w_0w_1 \dots w_n$ with $w_i \in \{0, 1\}$, for $0 \leq i \leq n$, we denote with $w[i]$ the symbol $w_i$ at position $i$ in $w$. Let $|w| = n$ be the length of $w$. We write the concatenation of strings $u$ and $v$ as $u \cdot v$. For non-negative integers $i \leq j$ we denote with $[i, j]$ the interval of natural numbers $n$ with $i \leq n \leq j$. 
An edge periodic temporal graph, $G = (V, E, \tau)$ (see also \cite{erlebach2020game}) consists of a graph $G = (V, E)$ called the underlying graph and a function $\tau : E \rightarrow \{0, 1\}$ where $\tau$ maps each edge $e$ to a string $\tau(e) \in \{0, 1\}$
such that $e$ exists in a time step $t \geq 0$ if and only if $\tau(e)[t] = 1$, where $\tau(e)[t] := \tau(e)[t~ mod~ |\tau(e)|]$.
Every edge $e$ exists in at least one time step, that is, for each edge $e$ there is some $t_e \in [0, |\tau(e)| - 1]$ with $\tau(e)[t_e] = 1$. We might abbreviate $i$ repetitions of the same symbol $\sigma$ in $\tau(e)$ as $\sigma^i$.
We denote with $G(t)$ the subgraph of $G$ present in time step $t$. We do not assume that $G$ is connected in any time step. If not stated otherwise, we assume an edge periodic graph to be undirected.


\section{State of the art}