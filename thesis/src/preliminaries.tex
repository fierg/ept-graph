\chapter{Preliminaries}
\label{ch:Preliminaries}
%% ==============================

\section{Notation}
We denote by $\mathbb{N}$ the set of non-negative integers $\{0, 1, 2, \dots \}$. For a word $w = w_1 w_2 \dots w_n$ with $w_i \in \sum$ for $1 \leq i \leq n$, $w[i]$ represents the symbol $w_i$ at position $i$ in $w$ and we denote by $w_R = w_n \dots w_2 w_1$ the reverse of $w$. Moreover, for every $\sigma \in \Sigma$, we denote by $\#\sigma(w)$ the number of times the letter $\sigma$ appears in $w$. Concatenation of strings $u$ and $v$ is written as $u \cdot v$. 

A natural number $n > 1$ is called \textit{composite} if it is the product of two smaller numbers, otherwise we say that $n$ is \textit{prime}. Two integers $m, n \in N$ are called \textit{co-prime} if their greatest common divisor is 1.


\textit{A deterministic finite automaton} ($\DFA$) is a 5-tuple $A = \lbrace\Sigma, Q, q_I , \sigma, F\rbrace$, where $\Sigma$ is a finite non-empty alphabet, $Q$ is a finite set of states, $\sigma : Q \times \Sigma \rightarrow Q$ is a transition function, $q_I \in Q$ is the initial state, and $F \subseteq Q$ is a set of accepting states. The states in $Q \backslash F$ are called rejecting states. The transition function $\sigma$ is expanded to words by defining it recursively $\sigma : Q \times \Sigma^* \rightarrow Q$ is $\sigma(q, \epsilon) = q$ and $\sigma(q, w_1, w_2 \dots w_n ) = \sigma(\sigma(q, w_1 w_2 \dots w_{n-1} ), w_n )$. The \textit{run} of $A$ on a word $w = w_1 \dots w_n$ is the sequence of states $s_0 , s_1 , \dots , s_n$ such that $s_0 = q_I$ and for each $1 \leq i \leq n$ it holds that $\sigma(s_{i-1} , w_i ) = s_i$ . Note that $s_n = \sigma(q_I , w)$. The $\DFA$ A \textit{accepts} $w$ iff $\sigma(q_I , w) \in F$. Otherwise, $A$ \textit{rejects} $w$. The set of words accepted by $A$ is denoted $L(A)$ and is called the \textit{language} of $A$. A language accepted by some $\DFA$ is called a \textit{regular language}. We refer to the size of a $\DFA$ $A$, denoted $|A|$, as the number of states in $A$. A $\DFA$ $A$ is \textit{minimal} if every $\DFA B$ such that $L(B) = L(A)$ satisfies $|B| \geq |A|$.

We call a $\DFA$ $A$ \textit{composite} if there exists a family $(B_i)_{1 \leq i \leq k}$ of DFAs with $|B_i| < |A|$ for all $1 \leq i \leq k$ such that $L(A) = \cup_{1\leq i \leq k} L(B_i)$ and call the family $(B_i)_{1\leq i \leq k}$ a \textit{decomposition} of $A$. Note that, all $B_i$ in the decomposition satisfy $|B_i| < |A|$ and $L(A) \subseteq L(B_i)$. Such DFAs are called \textit{factors} of $A$, and $(B_i)_{1\leq i \leq k}$ is also called a $k$-factor decomposition of $A$. The
width of $A$ is the smallest $k$ for which there is a $k$-factor decomposition of $A$, and we say that $A$ is $k$-factor composite iff $width(A) \leq k$. We call a DFA $A$ prime if it is not composite.

An edge periodic temporal graph, $G = (V, E, \tau)$ (see also \cite{erlebach2020game}) consists of a graph $G = (V, E)$ called the underlying graph and a function $\tau : E \rightarrow \{0, 1\}$ where $\tau$ maps each edge $e$ to a string $\tau(e) \in \{0, 1\}^*$, such that $e$ exists in a time step $t \geq 0$ if and only if $\tau(e)[t] = 1$, where $\tau(e)[t] := \tau(e)[t~ mod~ |\tau(e)|]$. We denote the subgraph of $G$ in time step $t$ with $G(t)$.


\section{State of the art}
The study of temporal graphs, also known as time-varying or dynamic graphs, has gained significant attention in recent years. Researchers have been actively exploring various aspects of temporal graphs, including their representation \cite{Holme_2012}, analysis \cite{DBLP:journals/corr/Erlebach0K15}\cite{DBLP:journals/corr/Michail15}, and mining techniques. Temporal graph mining focuses on extracting valuable information from temporal graphs. Researchers have developed techniques for tasks such as community detection, which aims to identify groups of nodes that exhibit cohesive patterns of interactions over time (e.g. graph minors or subgraphs) \cite{arrighi2022multiparameter}. Link prediction techniques have also been explored to predict future edge occurrences based on historical data \cite{temporalLinkPrediction}. Additionally, various metrics have been proposed to quantify important properties of temporal graphs, such as temporal centrality and clustering coefficients \cite{temporalClusterCoefficient}. Understanding the evolution and dynamics of temporal graphs is another research area. Models have been developed to simulate the evolution of temporal graphs, considering factors such as edge arrival, departure, and temporal dependencies. The study of influence and cascading behavior in temporal graphs explores how information or influence spreads over time, including modeling cascades and studying diffusion processes \cite{temporalEvolution}. Temporal network analysis involves the development of measures and techniques to characterize and visualize temporal graphs. Various metrics have been proposed to capture properties like assortativity, modularity, and temporal path analysis. Visualization techniques have been designed to effectively represent and explore the dynamic nature of temporal graphs \cite{kerracher2014design}. Temporal graph research finds applications in diverse fields. For example, it has been applied to study social networks, analyzing the evolution of friendships, communities, and information diffusion. In transportation systems, temporal graphs help model traffic patterns, optimize routes, and analyze network dynamics \cite{tang2009temporal}. In the realm of biology, temporal graphs have been used to study protein-protein interaction networks \cite{fu2022dppin}, gene regulatory networks, and other biological systems to understand dynamic processes \cite{dibrita2022temporal}. Overall, the study of temporal graphs is a multidisciplinary field, with researchers from computer science, network science, physics, and other domains contributing to its advancements. Ongoing research continues to explore new techniques and methodologies for analyzing, modeling, and understanding the complex dynamics of temporal graphs.

In contrary, edge periodic graphs is not a widely known term and is not as well established as temporal graphs, especially the transformation of temporal graphs into edge periodic graphs is not explored. On the other hand, there are various resources regarding the study of prime languages and unary prime languages and the deterministic finite automatons recognizing these, see \cite{prime-languages},\cite{unara-prime-languages},\cite{DBLP:journals/corr/abs-2107-04683}. The proofs are formalized for DFAs but they are identical with binary labels as transitions are trivial and either 1's or 0's can represent the final states $Q_f$. 

\section{Real World data (F2F Graphs)}
Some real world data is helpful in evaluating the effectiveness of the decomposition, therefore a collection of temporal graphs has been gathered from this Stanford \href{https://snap.stanford.edu/data/comm-f2f-Resistance.html}{dataset}. These so called \textit{dynamic face-to-face interaction networks} represent the interactions that happen during discussions between a group of participants playing the Resistance game. This dataset contains networks extracted from 62 games. Each game is played by 5-8 participants and lasts between 45-60 minutes. They extracted dynamically evolving networks from the free-form discussions using the ICAF algorithm~\cite{f2f-bai2019predicting}. The extracted networks are used to characterize and detect group deceptive behavior using the DeceptionRank algorithm~\cite{f2f-kumar2021deception}. The networks are unweighted, directed and temporal. Each node represents a participant. At each 1/3 second, a directed edge from node $u$ to $v$ indicates participant $u$ looks at participant $v$ (or the laptop).

\begin{figure}[h]
	\centering
		\begin{tikzpicture}[every edge quotes/.style = {auto, font=\footnotesize, sloped}]
		\begin{scope}[every node/.style={circle,fill}]
			\node (A) at (0,0) {};
			\node (B) at (0,4) {};
			\node (C) at (4,0) {};
			\node (D) at (4,4) {};
			\node (E) at (2,2) {};
		\end{scope}
		
		\draw (A) edge["101\dots"] (B)
		(D) edge["001\dots"] (C)
		(A) edge["000\dots"] (C)
		(D) edge[""] (E)
		(B) edge["011\dots"] (D)
		(B) edge[""] (E)
		(A) edge[""] (E)	
		(C) edge[""] (E)
		(B) edge[""] [loop] (B)
		(A) edge[""] [loop left] (A)
		(C) edge[""] [loop right] (C)	
		(D) edge[""] [loop] (D);		
		\end{tikzpicture}
	
	\label{tikz:realWorldDataFiles}
	\caption{Example of the real word input data (F2F Files)}
\end{figure}

There are also weighted versions of the networks for other purposes and more of a social analysis but we focus on the unweighted versions as they fit the definition of temporal graphs best.