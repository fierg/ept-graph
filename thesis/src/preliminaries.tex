\chapter{Preliminaries}
\label{ch:Preliminaries}
%% ==============================

\section{Notation}
For a string $w = w_0w_1 \dots w_n$ with $w_i \in \{0, 1\}$, for $0 \leq i \leq n$, $w[i]$ represents the symbol $w_i$ at position $i$ in $w$. Let the length of $w$ be $|w| = n$. Concatenation of strings $u$ and $v$ is written as $u \cdot v$. An edge periodic temporal graph, $G = (V, E, \tau)$ (see also \cite{erlebach2020game}) consists of a graph $G = (V, E)$ called the underlying graph and a function $\tau : E \rightarrow \{0, 1\}$ where $\tau$ maps each edge $e$ to a string $\tau(e) \in \{0, 1\}$, such that $e$ exists in a time step $t \geq 0$ if and only if $\tau(e)[t] = 1$, where $\tau(e)[t] := \tau(e)[t~ mod~ |\tau(e)|]$. Every edge $e$ exists in at least one time step, that is, for each edge $e$ there is some $t_e \in [0, |\tau(e)| - 1]$ with $\tau(e)[t_e] = 1$. We abbreviate $i$ repetitions of identical symbols $\sigma$ in $\tau(e)$ as $\sigma^i$. We denote the subgraph of $G$ in time step $t$ with $G(t)$.


\section{State of the art}

The study of temporal graphs, also known as time-varying or dynamic graphs, has gained significant attention in recent years. Researchers have been actively exploring various aspects of temporal graphs, including their representation \cite{Holme_2012}, analysis \cite{DBLP:journals/corr/Erlebach0K15}\cite{DBLP:journals/corr/Michail15}, and mining techniques. Temporal graph mining focuses on extracting valuable information from temporal graphs. Researchers have developed techniques for tasks such as community detection, which aims to identify groups of nodes that exhibit cohesive patterns of interactions over time (e.g. graph minors or subgraphs) \cite{arrighi2022multiparameter}. Link prediction techniques have also been explored to predict future edge occurrences based on historical data \cite{temporalLinkPrediction}. Additionally, various metrics have been proposed to quantify important properties of temporal graphs, such as temporal centrality and clustering coefficients \cite{temporalClusterCoefficient}. Understanding the evolution and dynamics of temporal graphs is another research area. Models have been developed to simulate the evolution of temporal graphs, considering factors such as edge arrival, departure, and temporal dependencies. The study of influence and cascading behavior in temporal graphs explores how information or influence spreads over time, including modeling cascades and studying diffusion processes \cite{temporalEvolution}. Temporal network analysis involves the development of measures and techniques to characterize and visualize temporal graphs. Various metrics have been proposed to capture properties like assortativity, modularity, and temporal path analysis. Visualization techniques have been designed to effectively represent and explore the dynamic nature of temporal graphs \cite{kerracher2014design}. Temporal graph research finds applications in diverse fields. For example, it has been applied to study social networks, analyzing the evolution of friendships, communities, and information diffusion. In transportation systems, temporal graphs help model traffic patterns, optimize routes, and analyze network dynamics \cite{tang2009temporal}. In the realm of biology, temporal graphs have been used to study protein-protein interaction networks \cite{fu2022dppin}, gene regulatory networks, and other biological systems to understand dynamic processes \cite{dibrita2022temporal}. Overall, the study of temporal graphs is a multidisciplinary field, with researchers from computer science, network science, physics, and other domains contributing to its advancements. Ongoing research continues to explore new techniques and methodologies for analyzing, modeling, and understanding the complex dynamics of temporal graphs.

In contrary, edge periodic graphs is not a widely known term and is not as well established as temporal graphs, especially the transformation of temporal graphs into edge periodic graphs is not explored.

\section{Real World data (F2F Graphs)}


\begin{figure}[h]
	\centering
		\begin{tikzpicture}[every edge quotes/.style = {auto, font=\footnotesize, sloped}]
		\begin{scope}[every node/.style={circle,fill}]
			\node (A) at (0,0) {};
			\node (B) at (0,4) {};
			\node (C) at (4,0) {};
			\node (D) at (4,4) {};
			\node (E) at (2,2) {};
		\end{scope}
		
		\draw (A) edge["101\dots"] (B)
		(D) edge["001\dots"] (C)
		(A) edge["000\dots"] (C)
		(D) edge[""] (E)
		(B) edge["011\dots"] (D)
		(B) edge[""] (E)
		(A) edge[""] (E)	
		(C) edge[""] (E)
		(B) edge[""] [loop] (B)
		(A) edge[""] [loop left] (A)
		(C) edge[""] [loop right] (C)	
		(D) edge[""] [loop] (D);		
		\end{tikzpicture}
	
	\label{tikz:realWorldDataFiles}
	\caption{Example of the real word input data (F2F Files)}
\end{figure}