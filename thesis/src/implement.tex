\chapter{Implementation}
\label{ch:Implementation}
%% ==============================
All algorithms described have been implemented using Kotlin, because it can be compiled for the Java Virtual Machine and for native execution, therefore it seemed like a good balance between a native implementation and higher language conciseness and fault tolerance. The project is realized as a maven project, to simplify dependency management.

\section{File Reader}
\label{ch:Implementation:file-reader}
TODO: f2f file reader

\section{Decomposition using Maximal Divisors}
\label{ch:Implementation:max-divisor}
For the initial implementation, only maximal divisors were used as this is already well studied. For an input array of size $n$, all maximal divisors are lazily yielded into a sequence, via a \textit{NumbersExtensions}. Extension functions are a useful Kotlin tool if you want to extend the static functionality of base classes, e.g. Integers. Then a preliminary Factor is created for each maximal divisor, with an empty array the size of the divisor and a job is spawned for each factor as the periodicity of the factors can be tested in parallel. With the help of Kotlin Co-routines, a modest multi-threading approach is implemented where the main algorithm thread waits for all the factors to be checked. Afterwards, they are collected into a \textit{Cover} object, which handles all the validation and quality assessment. For this, a cover array is created, where all factors are consolidated and uncovered values so called outliers are visible. As not all inputs are decomposable and by default we do not allow  $|B_i| = |A|$, we may find labels for which the \DFAs are not decomposable. In this case, we consider them partially decomposable, as the found factors create a cover array which is partially identical to the original but not in all places, so a warning is issued with the amount of hard outliers.

\section{Greedy Short Factor Decomposition}
\label{ch:Implementation:greedy-short-factor}
The implementation of the greedy approach to find shorter factors, is very similar to the previous method. As described in Section \ref{ch:novel-algos:greedy-short-factors}, all divisors of the length or the original \DFA are considered instead of just the maximal divisors. Multi-threading in the job implementation is even more useful as we consider a greater amount of possible factor sizes. All found factors are again collected into a Cover object, this time we are only adding factors to the cover object if their addition changes the resulting cover array. In the example from the Figure \ref{fig:short-factors} we would only add $A_2$ and skip over $A_4$ since the further addition of factors does not change the resulting decomposition. If after an addition of a factor a complete decomposition is formed with $L(A) = \cup_{B_i \in \text{Cover}} L(B_i)$ which is validated by counting the outliers and checking for $out(\{B_i \in~\text{Cover}\}) = 0$. Since is to be expected that a considerable amount of \DFAs generated from labels from real world data, a decomposition with a given $precision(\{B_i \in~\text{Cover}\})$ is also implemented.

\section{Fourier Transformed Decomposition}
\label{ch:Implementation:fourier-transform}
With the goal of increased explainability, the Fourier-transformed decomposition is implemented according to Section \ref{ch:novel-algos:fourier}. Similar to the greedy approach, all divisors of the length or the original \DFA are considered. With the Fourier-transformation of the possible factors, we first remove all multiples of periods as described. Afterwards, the single set value requirement $|F_{B_i}| = 1$ is enforced, by replacing all factors not meeting the requirement, by multiple factors which do hold. The collecting into a cover is performed greedily again, to reduce the number of factors in the cover to improve explainability.

Independent from the decomposition method used, a \textit{Cover} object can be turned into an EPT multi-graph with the same nodes as the original f2f graph but for each cover there is now a multi-edge, one for each decomposition with the cover array as label.

\section{External Libraries}
\label{ch:Implementation:external-libs}
Some external libraries are used throughout this project which was quite simple through the use of maven. \emph{org.junit.jupiter} provided the packages \emph{junit}, \emph{engine} and \emph{api} in version 4.13.2, which allows a simple test unit creation and execution. Multi-threading was enabled by use of Kotlin \emph{coroutines} in version 1.7.1, provided by Jetbrains.

TODO: jgraph, lets plot, argparser, 

\section{Usage}
\label{ch:Implementation:usage}
To enable a convenient usage, the algorithm is obtainable as jar file but it can also be built from sources. It provides a simple command line interface

\verb|Usage: EPT Graph Reader options_list|\\
\verb|Arguments: |\\
\verb|input -> input (Network id in range (0..61) { Int }|\\
\verb|threshold [1.0] -> Min threshold of cover to be valid (optional) { Double }|\\
\verb|Options: |\\
\verb|--dotenv, -env [false] -> Use config from .env file|\\
\verb|		(Recommended usage due to amount of args) |\\
\verb|--state, -s [false] -> Invert state to substitute in decomposition|\\
\verb|		(if set, decomposition will replace 0s instead of 1s) |\\
\verb|--Mode of decomposing DFAs [GREEDY_SHORT_FACTORS] -> Choose how|\\
\verb|		a decomposition is found, using only maximal divisors,|\\
\verb|		using all factors and greedily collect up to the threshold|\\
\verb|		or perform a fourier transform for increased understandability.|\\
\verb|		{ Value should be one of [max_divisors, greedy_short_factors,|\\
\verb| 		fourier_transform] }|\\
\verb|--Mode of composing factors [OR] -> Choose how a composition is formed,|\\
\verb|		using AND or OR operator for adding factors together to|\\
\verb|		a decomposition. { Value should be one of [and, or] }|\\
\verb|--skipSelfEdges, -skipSelfEdges [false] -> Skip loop back edges|\\
\verb|		with same source and target, these are often useless. |\\
\verb|--debug, -d [false] -> Turn on debug mode |\\
\verb|--quiet, -q [false] -> Turn on quiet mode |\\
\verb|--deltaWindowPreprocessing [0] -> Delta window|\\
\verb|		 in preprocessing of the label { Int }|\\
\verb|--deltaWindowAlgo [0] -> Delta window during decomposing { Int }|\\
\verb|--help, -h -> Usage info |\\

TODO: describe, .env file

TODO: github link