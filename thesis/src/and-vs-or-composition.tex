\chapter{AND vs OR Decomposition}
\label{chap:and-vs-or}
All theoretical studies so far used the \andDecomp with the complement over all factors.
This idea is easily expandable to the \orDecomp, where the union over all factors $B_i$ results in the original language, $L(A) = \cup_{1\leq i \leq k} L(B_i)$.
Note that, all $B_i$ in the decomposition satisfy $|B_i| < |A|$ and $L(A) \supseteq L(B_i)$.
With this change, a single factor implies certain final states, whereas, with the \andDecomp, all factors have to be considered to know whether a given state is final in $A$.
As we defined the \orDecomp as a conceptual counterpart to the \andDecomp, we considered different \DFAs with either a \andDecomp or an \orDecomp.
Of course, there exist \DFAs with both and- and \orDecomp but we also prove that this is not the case for every \DFA.
For an example see Figures \ref{fig:factors-and-example} and \ref{fig:factors-or-example}, where $A$ has a \andDecomp and $B$ has a \orDecomp but not vice versa.

\begin{figure}[h]
	\begin{minipage}[t]{\textwidth}
		\centering
		\begin{tikzpicture}[shorten >=1pt,node distance=2cm,on grid,auto] 
			\node[state,initial] (q_0)   {$q_0$}; 
			\node[state,accepting] (q_1) [ right=of q_0] {$q_1$}; 
			\node[state] (q_2) [ right=of q_1] {$q_2$}; 
			\node[state,accepting] (q_3) [ right=of q_2] {$q_3$};
			\node[state](q_4) [ right=of q_3] {$q_4$};
			\node[state](q_5) [ right=of q_4] {$q_5$};
			\path[->] 
			(q_0) edge  node {} (q_1)
			(q_1) edge  node {} (q_2)
			(q_2) edge  node {} (q_3)
			(q_3) edge  node {} (q_4)
			(q_4) edge  node {} (q_5)
			(q_5) edge[bend right, above]  node {} (q_0);
		\end{tikzpicture}		
	\end{minipage}
	\begin{minipage}[b]{0.39\textwidth}
		\centering
		\begin{tikzpicture}[shorten >=1pt,node distance=2cm,on grid,auto] 
			\node[state,initial] (q_0)   {$q_0$}; 
			\node[state,accepting] (q_1) [right=of q_0] {$q_1$}; 
			\path[->] 
			(q_0) edge  node {} (q_1)
			(q_1) edge[bend right, above]  node {} (q_0);
		\end{tikzpicture}
	\end{minipage}
	\begin{minipage}[b]{0.59\textwidth}
		\centering
		\begin{tikzpicture}[shorten >=1pt,node distance=2cm,on grid,auto] 
			\node[state,initial,accepting] (q_0)   {$q_0$}; 
			\node[state,accepting] (q_1) [right=of q_0] {$q_1$}; 
			\node[state](q_2) [right=of q_1] {$q_2$};
			\path[->] 
			(q_0) edge  node {} (q_1)
			(q_1) edge  node {} (q_2)
			(q_2) edge[bend right, above]  node {} (q_0);
			
		\end{tikzpicture}
	\end{minipage}
	\caption{The DFA $A$ and its factor $A_2$ and $A_3$ (AND-decomposition)}
	\label{fig:factors-and-example}
\end{figure}

\begin{figure}[h]
	\begin{minipage}[t]{\textwidth}
		\centering
		\begin{tikzpicture}[shorten >=1pt,node distance=2cm,on grid,auto] 
			\node[state,initial] (q_0)   {$q_0$}; 
			\node[state,accepting] (q_1) [ right=of q_0] {$q_1$}; 
			\node[state] (q_2) [ right=of q_1] {$q_2$}; 
			\node[state,accepting] (q_3) [ right=of q_2] {$q_3$};
			\node[state,accepting](q_4) [ right=of q_3] {$q_4$};
			\node[state,accepting] (q_5) [ right=of q_4] {$q_5$};
			\path[->] 
			(q_0) edge  node {} (q_1)
			(q_1) edge  node {} (q_2)
			(q_2) edge  node {} (q_3)
			(q_3) edge  node {} (q_4)
			(q_4) edge  node {} (q_5)
			(q_5) edge[bend right, above]  node {} (q_0);
		\end{tikzpicture}		
	\end{minipage}
	\begin{minipage}[b]{0.39\textwidth}
		\centering
		\begin{tikzpicture}[shorten >=1pt,node distance=2cm,on grid,auto] 
			\node[state,initial] (q_0)   {$q_0$}; 
			\node[state,accepting] (q_1) [right=of q_0] {$q_1$}; 
			\path[->] 
			(q_0) edge  node {} (q_1)
			(q_1) edge[bend right, above]  node {} (q_0);
			
		\end{tikzpicture}
	\end{minipage}
	\begin{minipage}[b]{0.59\textwidth}
		\centering
		\begin{tikzpicture}[shorten >=1pt,node distance=2cm,on grid,auto] 
			\node[state,initial] (q_0)   {$q_0$}; 
			\node[state,accepting] (q_1) [right=of q_0] {$q_1$}; 
			\node[state](q_2) [right=of q_1] {$q_2$};
			\path[->] 
			(q_0) edge  node {} (q_1)
			(q_1) edge  node {} (q_2)
			(q_2) edge[bend right, above]  node {} (q_0);
			
		\end{tikzpicture}
	\end{minipage}
	\caption{The DFA $B$ and its factors $B_2$ \& $B_3$ (OR-decomposition)}
	\label{fig:factors-or-example}
\end{figure}

To prove that there exist words that are and-decomposable but not or-decomposable and vice versa, we need a bit of background.
For this purpose we have extended the logic operators $\lor$ and $\land$ to binary words.

\begin{theorem}
	There exists a word $w \in \{0,1\}^*$ such that $w$ is and-decomposable but not or-decomposable
\end{theorem}

\begin{proof}
	Assume $\overbar{w}$ is not and-decomposable and $w = v_1 \lor v_2 \lor \dots \lor v_k$.
	Using the fact that $A \lor B = \overbar{\overbar{A} \land \overbar{B}}$, we obtain $w = \overbar{\overbar{v_1} \land \overbar{v_2} \land \dots \land \overbar{v_k}}$ which is equal to $\overbar{w} = \overbar{v_1} \land \overbar{v_2} \land \dots \land \overbar{v_k}$.
	This implies $w$ is not or-decomposable via contradiction.
\end{proof}


\begin{theorem}
	There exists a word $w$ such that $w$ is or decomposable but not and-decomposable
\end{theorem}

\begin{proof}
	Assume $\overbar{w}$ is not or-decomposable and $w = v_1 \land v_2 \land \dots \land v_k$.
	Using the fact  that $A \land B = \overbar{\overbar{A} \lor \overbar{B}}$, we obtain $w = \overbar{\overbar{v_1} \lor \overbar{v_2} \lor \dots \lor \overbar{v_k}}$ which is equal to $\overbar{w} = \overbar{v_1} \lor \overbar{v_2} \lor \dots \lor \overbar{v_k}$.
	This implies $w$ is not and-decomposable via contradiction.
\end{proof}

\begin{corollary}
	The set of or-decomposable strings and the set of and-decomposable string is not comparable.
\end{corollary}

As we have now proven that the results of \andDecomp and \orDecomp are not comparable, we will will focus on \orDecomp for explainability purposes, but still evaluate the effectiveness of both methods.


%%% Local Variables: 
%%% mode: latex
%%% TeX-master: "thesis"
%%% End: 