\chapter{AND vs OR Decomposition}
\label{chap:and-vs-or}
All theoretical studies so far used the \andDecomp with the complement over all factors.
This idea is easily expandable to the \orDecomp, where the union over all factors $B_i$ results in the original language, $L(A) = \cup_{1\leq i \leq k} L(B_i)$.
Note that, all $B_i$ in the decomposition satisfy $|B_i| < |A|$ and $L(A) \supseteq L(B_i)$.
With this change, a single factor implies certain final states, whereas, with the \andDecomp, all factors have to be considered to know whether a given state is final in $A$.
As we defined the \orDecomp as a conceptual counterpart to the \andDecomp, we considered different \DFAs with either an \andDecomp or an \orDecomp.
Of course, there exist \DFAs with both and- and \orDecomp but we also prove that this is not the case for every \DFA.
For an example see Figures \ref{fig:factors-and-example} and \ref{fig:factors-or-example}, where $A$ has a \andDecomp and $B$ has a \orDecomp but not vice versa.

\begin{figure}[h]
	\begin{minipage}[t]{\textwidth}
		\centering
		\begin{tikzpicture}[shorten >=1pt,node distance=2cm,on grid,auto] 
			\node[state,initial] (q_0)   {$q_0$}; 
			\node[state,accepting,fill=lightgray,fill opacity=0.5] (q_1) [ right=of q_0] {$q_1$}; 
			\node[state] (q_2) [ right=of q_1] {$q_2$}; 
			\node[state,accepting,fill=lightgray,fill opacity=0.5] (q_3) [ right=of q_2] {$q_3$};
			\node[state](q_4) [ right=of q_3] {$q_4$};
			\node[state](q_5) [ right=of q_4] {$q_5$};
			\path[->] 
			(q_0) edge  node {} (q_1)
			(q_1) edge  node {} (q_2)
			(q_2) edge  node {} (q_3)
			(q_3) edge  node {} (q_4)
			(q_4) edge  node {} (q_5)
			(q_5) edge[bend right, above]  node {} (q_0);
		\end{tikzpicture}		
	\end{minipage}
	\begin{minipage}[b]{0.39\textwidth}
		\centering
		\begin{tikzpicture}[shorten >=1pt,node distance=2cm,on grid,auto] 
			\node[state,initial] (q_0)   {$q_0$}; 
			\node[state,accepting,fill=lightgray,fill opacity=0.5] (q_1) [right=of q_0] {$q_1$}; 
			\path[->] 
			(q_0) edge  node {} (q_1)
			(q_1) edge[bend right, above]  node {} (q_0);
		\end{tikzpicture}
	\end{minipage}
	\begin{minipage}[b]{0.59\textwidth}
		\centering
		\begin{tikzpicture}[shorten >=1pt,node distance=2cm,on grid,auto] 
			\node[state,initial,accepting,fill=lightgray,fill opacity=0.5] (q_0)   {$q_0$}; 
			\node[state,accepting,fill=lightgray,fill opacity=0.5] (q_1) [right=of q_0] {$q_1$}; 
			\node[state](q_2) [right=of q_1] {$q_2$};
			\path[->] 
			(q_0) edge  node {} (q_1)
			(q_1) edge  node {} (q_2)
			(q_2) edge[bend right, above]  node {} (q_0);
			
		\end{tikzpicture}
	\end{minipage}
	\caption{The DFA $A$ and its factor $A_2$ and $A_3$ (AND-decomposition)}
	\label{fig:factors-and-example}
\end{figure}

\begin{figure}[h]
	\begin{minipage}[t]{\textwidth}
		\centering
		\begin{tikzpicture}[shorten >=1pt,node distance=2cm,on grid,auto] 
			\node[state,initial] (q_0)   {$q_0$}; 
			\node[state,accepting,fill=lightgray,fill opacity=0.5] (q_1) [ right=of q_0] {$q_1$}; 
			\node[state] (q_2) [ right=of q_1] {$q_2$}; 
			\node[state,accepting,fill=lightgray,fill opacity=0.5] (q_3) [ right=of q_2] {$q_3$};
			\node[state,accepting,fill=lightgray,fill opacity=0.5](q_4) [ right=of q_3] {$q_4$};
			\node[state,accepting,fill=lightgray,fill opacity=0.5] (q_5) [ right=of q_4] {$q_5$};
			\path[->] 
			(q_0) edge  node {} (q_1)
			(q_1) edge  node {} (q_2)
			(q_2) edge  node {} (q_3)
			(q_3) edge  node {} (q_4)
			(q_4) edge  node {} (q_5)
			(q_5) edge[bend right, above]  node {} (q_0);
		\end{tikzpicture}		
	\end{minipage}
	\begin{minipage}[b]{0.39\textwidth}
		\centering
		\begin{tikzpicture}[shorten >=1pt,node distance=2cm,on grid,auto] 
			\node[state,initial] (q_0)   {$q_0$}; 
			\node[state,accepting,fill=lightgray,fill opacity=0.5] (q_1) [right=of q_0] {$q_1$}; 
			\path[->] 
			(q_0) edge  node {} (q_1)
			(q_1) edge[bend right, above]  node {} (q_0);
			
		\end{tikzpicture}
	\end{minipage}
	\begin{minipage}[b]{0.59\textwidth}
		\centering
		\begin{tikzpicture}[shorten >=1pt,node distance=2cm,on grid,auto] 
			\node[state,initial] (q_0)   {$q_0$}; 
			\node[state,accepting,fill=lightgray,fill opacity=0.5] (q_1) [right=of q_0] {$q_1$}; 
			\node[state](q_2) [right=of q_1] {$q_2$};
			\path[->] 
			(q_0) edge  node {} (q_1)
			(q_1) edge  node {} (q_2)
			(q_2) edge[bend right, above]  node {} (q_0);
			
		\end{tikzpicture}
	\end{minipage}
	\caption{The DFA $B$ and its factors $B_2$ \& $B_3$ (OR-decomposition)}
	\label{fig:factors-or-example}
\end{figure}

To prove that there exist words that are and-decomposable but not OR-decomposable and vice versa, we need a bit of background.
For this purpose we have extended the logic operators $\lor$ and $\land$ to binary words.
We also need the following Lemma:
\begin{lemma}
	Consider $w$, a binary string and $A$ the corresponding unary \DFA.
	Let the complement of $w$ be $w'$ where every value is flipped and $A'$ its corresponding \DFA.
	An \andDecomp of $A'$ is equivalent to an \orDecomp of $A$.
\end{lemma}

\begin{proof}
	Since an \andDecomp covers the non final states of a \DFA and an \orDecomp covers the final states, an \andDecomp of the complement of a \DFA will cover its final states, exactly like an \orDecomp and vice versa.
\end{proof}

\begin{theorem}
	There exists a word $w \in \{0,1\}^*$ such that $w$ is OR-decomposable but not AND-decomposable
\end{theorem}

\begin{proof}
	Assume the \DFA $B$ from Figure \ref{fig:factors-or-example}.
	As from Lemma 7 from \cite{unara-prime-languages}, for an \andDecomp it suffices to consider the quotient \DFAs of $B$.
	This implies that there exists no \andDecomp for \DFA $B$ since 2 and 3 are the only divisors and the possible factors of this size, do not generate a valid \andDecomp.
\end{proof}

\begin{theorem}
	There exists a word $w$ such that $w$ is AND-decomposable but not OR-decomposable
\end{theorem}

\begin{proof}
	Let $A$ be the \DFA from Figure \ref{fig:factors-and-example}.
	Consider the complement \DFA of $A$, $A'$ with $F'=\lbrace q_0, q_2, q_4, q_5\rbrace$.
	\begin{figure}[h]
			\centering
			\begin{tikzpicture}[shorten >=1pt,node distance=2cm,on grid,auto] 
				\node[state,initial,accepting,fill=lightgray,fill opacity=0.5] (q_0)   {$q_0$}; 
				\node[state] (q_1) [ right=of q_0] {$q_1$}; 
				\node[state,accepting,fill=lightgray,fill opacity=0.5] (q_2) [ right=of q_1] {$q_2$}; 
				\node[state] (q_3) [ right=of q_2] {$q_3$};
				\node[state,accepting,fill=lightgray,fill opacity=0.5](q_4) [ right=of q_3] {$q_4$};
				\node[state,accepting,fill=lightgray,fill opacity=0.5](q_5) [ right=of q_4] {$q_5$};
				\path[->] 
				(q_0) edge  node {} (q_1)
				(q_1) edge  node {} (q_2)
				(q_2) edge  node {} (q_3)
				(q_3) edge  node {} (q_4)
				(q_4) edge  node {} (q_5)
				(q_5) edge[bend right, above]  node {} (q_0);
			\end{tikzpicture}	
		\label{fig:a-complement-from-proof}
		\caption{The DFA $A'$, complement of $A$}	
	\end{figure}
	Since the \DFA has 6 states, it has only 2 potential quotients with size 2 and 3.
	The non final states $q_1$ and $q_3$ are neither periodic in the length 2 or 3, so both quotients are empty as they consist only of final states.
	With both quotients empty, there is no valid \andDecomp of the \DFA $A'$ and therefore $A$ does not have a valid \orDecomp.
\end{proof}

\begin{corollary}
	The set of OR-decomposable strings and the set of AND-decomposable string is not comparable.
\end{corollary}

As we have now proven that the results of \andDecomp and \orDecomp are not comparable, we will will focus on \orDecomp for explainability purposes, but still evaluate the effectiveness of both methods.


%%% Local Variables: 
%%% mode: latex
%%% TeX-master: "thesis"
%%% End: 