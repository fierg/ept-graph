\chapter{Novel Approaches}
\label{chap:novel-algos}
We now have a concept of what an \enquote{explainable} edge label looks like, so we propose new ideas for decomposing \DFAs such that the resulting decomposition has a good structure, periodicity and width, resulting in good labels, understandable for humans. Additionally, we want the algorithm able to handle outliers of a given decomposition, which might result in even shorter labels. If we e.g. have a decomposition where a small factor covers most of the original \DFA, but not all of it, we need either a larger factor or remember the uncovered values, which could increase explainability. Additionally, all theoretical results so far used the \textit{and-composite} with the complement over all factors. This idea is easy expandable to the \textit{or-composite}, where the union over all factors $B_i$ results in the original language, $L(A) = \cup_{1\leq i \leq k} L(B_i)$. Note that, all $B_i$ in the decomposition satisfy $|B_i| < |A|$ and $L(A) \supseteq L(B_i)$. Which this change, a single factor implies certain final states, whereas with the composite, all factors have to be considered in order to know whether a given state $s$ is final in $A$. The implications of this change are considered in Chapter \ref{chap:and-vs-or}.

\section{Greedy Short Factors}
Searching for a decomposition using only the maximal divisors of $|Q|$, provides good solutions in \LogSpace, but there are certain limitations. Consider for example Figure \ref{fig:short-factors} depicting the trivial \DFA $A$ with its factors $A_2$ and $A_4$. Since for this \DFA, $|Q| = 8$, the only maximal divisor is 4 so the factor $A_4$ will be found be the original algorithm. In this particular example, there exists a smaller factor with size 2, $A_2$ which will only be found if checking all factors of $|Q|$ instead of only the maximal divisors.

\begin{figure}[h]
	\begin{minipage}[t]{\textwidth}
		\centering
		\begin{tikzpicture}[shorten >=1pt,node distance=1.5cm,on grid,auto] 
			\node[state,initial] (q_0)   {$q_0$}; 
			\node[state,accepting] (q_1) [ right=of q_0] {$q_1$}; 
			\node[state] (q_2) [ right=of q_1] {$q_2$}; 
			\node[state,accepting] (q_3) [ right=of q_2] {$q_3$};
			\node[state](q_4) [ right=of q_3] {$q_4$};
			\node[state,accepting] (q_5) [ right=of q_4] {$q_5$};
			\node[state] (q_6) [ right=of q_5] {$q_6$};
			\node[state,accepting] (q_7) [ right=of q_6] {$q_7$};
			\path[->] 
			(q_0) edge  node {} (q_1)
			(q_1) edge  node {} (q_2)
			(q_2) edge  node {} (q_3)
			(q_3) edge  node {} (q_4)
			(q_4) edge  node {} (q_5)
			(q_5) edge  node {} (q_6)
			(q_6) edge  node {} (q_7)
			(q_7) edge[bend right, above]  node {} (q_0);
		\end{tikzpicture}		
	\end{minipage}
	\begin{minipage}[b]{0.39\textwidth}
		\centering
		\begin{tikzpicture}[shorten >=1pt,node distance=1.5cm,on grid,auto] 
			\node[state,initial] (q_0)   {$q_0$}; 
			\node[state,accepting] (q_1) [right=of q_0] {$q_1$}; 
			\path[->] 
			(q_0) edge  node {} (q_1)
			(q_1) edge[bend right, above]  node {} (q_0);
		\end{tikzpicture}
	\end{minipage}
	\begin{minipage}[b]{0.59\textwidth}
		\centering
		\begin{tikzpicture}[shorten >=1pt,node distance=1.5cm,on grid,auto] 
			\node[state,initial] (q_0)   {$q_0$}; 
			\node[state,accepting] (q_1) [right=of q_0] {$q_1$}; 
			\node[state](q_2) [right=of q_1] {$q_2$};
			\node[state,accepting](q_3) [right=of q_2] {$q_3$};
			\path[->] 
			(q_0) edge  node {} (q_1)
			(q_1) edge  node {} (q_2)
			(q_2) edge  node {} (q_3)
			(q_3) edge[bend right, above]  node {} (q_0);
		\end{tikzpicture}
	\end{minipage}
	\caption{The DFA $A$ and its factors $A_4$ \& $A_2$}
	\label{fig:short-factors}
\end{figure}

Usually the algorithms and proofs are considering unary \DFAs consisting of a chain leading into a cycle of states. Since we obtain our \DFAs by transforming from a periodic label, we only have \DFAs with empty chains, therefore only considering unary permutation \DFA. This allows us to use a slightly simplified version of the Algorithm from \cite{DBLP:journals/corr/abs-2107-04683} as seen in Algorithm \ref{algo:composite} as we do not encounter unary automata with $\sigma(q, uv) \not = \sigma(q, vu)$. Additional we are not only interested in answering the yes/no question of the composite problem but we actually want to collect the factors and continue our computation.

\begin{algorithm}[H]
	\label{algo:composite}
	\DontPrintSemicolon
	\SetKwProg{Fn}{Function}{:}{}
	\Fn{getGreedyComposite($A = ⟨{a}, Q, qI , \sigma, F ⟩ $: unary DFA, integer k)}{
		CompositeList $\gets \emptyset$\; 
		\ForEach{factor $\in$ getAllFactors($A,k$)}{
			\If{factorChangesComposite($A$,binaryString)}{CompositeList.add(binaryString)}
		}
		\KwRet CompositeList\;
	}
	
	\Fn{getAllFactors($A = ⟨{a}, Q, qI , \sigma, F ⟩ $: unary DFA, integer k)}{
		FactorList $\gets \emptyset$\; 
		\ForEach{binaryString $\in \{0,1\}^{log|Q|}$ with $\leq k$ ones}{
			\If{isFactor($A$,binaryString)}{FactorList.add(binaryString)}
		}
		\KwRet FactorList\;
	}
	
	\Fn{isFactor($A = ⟨{a}, Q, qI , \sigma, F ⟩ $: unary DFA, binaryString)}{
		\ForEach{$q \in Q \setminus F$}{
			\If{not cover(A,q,binaryString)}{\KwRet false}
			\KwRet true
		}
	}
	
	\Fn{cover($A = ⟨{a}, Q, qI , \sigma, F ⟩ $: unary DFA, binaryString, $q \in Q \setminus F$)}{
		\ForEach{$i$ with wordCombination[i]=1}{
			$p_1 \gets i$'th prime divisor of $|Q|$\;
			\If{cover($A,q,\sigma(q,a^{|Q|/p_i})$)}{\KwRet true}
		}
		\KwRet false
	}
	
	\caption{Algorithm solving the Decomp problem for unary DFAs and returning a greedy composite from all factors.}
\end{algorithm}

TODO: explain algo and difference to original


The original problem was solvable in \LogSpace, but extending the problem to finding the shortest possible factors, we face a different problem. Finding all the factors is fast but choosing which factors are actually required in the decomposition, is not. This problem can be described as:

Given an input $u = 101101\dots$ as binary word, and integer $k$ and a set of words $w_1, w_2, \dots, w_n$ with $l(u) > l(w_n) ~\forall n$ and $l(w)$ is a strict divisor of $l(u) ~ \forall w$. Find indices $i_1, i_2 \dots i_k$ such that

\[
\underbrace{w_{i_1}, w_{i_1}, \dots, w_{i_1}}_{l(u)} \cup \dots \cup \underbrace{w_{i_k}, w_{i_k}, \dots, w_{i_k}}_{l(u)} = u
\]

To reduce the Set Cover problem to our novel problem, we'll create a instance as follows. For the universe $U$ we will construct a binary string $u$, where $u[i] = 1$ iff $i \in U$. For each subset $C_n$ of $U$ we create a word $w_n$ in the given set of words $w$ correspondingly. For each subset, there will be a word $w$, with $w[i] = 1$ iff $i \in C_n$. The Set Cover problem now involves finding the minimum number of subsets from $C$ that cover the entire universe $U$. The goal is to show that if we can efficiently solve the Set Cover problem for this constructed instance, we can efficiently solve the original problem. If the Set Cover problem finds a cover of size $k$, it means there is a set of indices $i_1, i_2, \dots, i_k$ that chooses from the words $w_i$ such that $w_{i_1}, w_{i_2}, \dots, w_{i_k}$ can form the same binary word as $u$, which satisfies the original problem. Conversely, if you can efficiently solve the original problem and find a set of indices $i_1, i_2, \dots, i_k$ that chooses from the words $w_i$ to form $u$, then the corresponding subsets $C_i$ in the constructed Set Cover instance cover the entire universe $U$. This means you've found a Set Cover of size $k$. This reduction demonstrates that solving the original problem is at least as hard as solving the Set Cover problem.

\section{Fourier Transform}
TODO: describe idea



%%% Local Variables: 
%%% mode: latex
%%% TeX-master: "thesis"
%%% End: 