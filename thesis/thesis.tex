\documentclass{wissdoc}
% Autor: Roland Bless 1996-2009, bless <at> kit.edu
% ----------------------------------------------------------------
% Diplomarbeit - Hauptdokument
% ----------------------------------------------------------------
%%
%% $Id: thesis.tex 65 2012-05-10 10:32:11Z bless $
%%
%
% Zum Erstellen zweiseitiger PDFs (für Buchdruck) in der Datei "wissdoc.cls" folgende Zeile abändern:
%
% \LoadClass[a4paper,12pt,oneside]{book} % diese Klasse basiert auf ``book''
% in
%\LoadClass[a4paper,12pt,titlepage]{book} % diese Klasse basiert auf ``book''
%
%
% wissdoc Optionen: draft, relaxed, pdf --> siehe wissdoc.cls
% ------------------------------------------------------------------
% Weitere packages: (Dokumentation dazu durch "latex <package>.dtx")
\usepackage[numbers,sort&compress]{natbib}
\usepackage{hyperref}
\usepackage{amsfonts}
\usepackage{csquotes}
\usepackage[ngerman,english]{babel}
\usepackage{tikz}
\usetikzlibrary{arrows}
\usetikzlibrary{arrows.meta}
\usetikzlibrary{positioning, quotes, automata, babel}
\usepackage{wrapfig}
\usepackage{pgfplots}
\usepackage{readarray}
\usepackage{changes}
\usepackage{amsmath}
\usepackage{mathtools}
\mathtoolsset{showonlyrefs}  
\usepackage{float}
\restylefloat{table}
\usepackage{environ}
\usepackage{caption}
\usepackage{subcaption}
\usepackage{xspace}
\usepackage[ruled,vlined]{algorithm2e}


\makeatletter
\newsavebox{\measure@tikzpicture}
\NewEnviron{scaletikzpicturetowidth}[1]{%
	\def\tikz@width{#1}%
	\def\tikzscale{1}\begin{lrbox}{\measure@tikzpicture}%
		\BODY
	\end{lrbox}%
	\pgfmathparse{#1/\wd\measure@tikzpicture}%
	\edef\tikzscale{\pgfmathresult}%
	\BODY
}
\makeatother



\tikzset{
	treenode/.style = {align=center, inner sep=0pt, text centered,
		font=\sffamily},
	arn_n/.style = {treenode, circle, white, font=\sffamily\bfseries, draw=black,
		fill=black, text width=1.5em},% arbre rouge noir, noeud noir
}

% \usepackage{varioref}
% \usepackage{verbatim}
% \usepackage{float}    %z.B. \floatstyle{ruled}\restylefloat{figure}
% \usepackage{subfigure}
% \usepackage{fancybox} % für schattierte,ovale Boxen etc.
% \usepackage{tabularx} % automatische Spaltenbreite
% \usepackage{supertab} % mehrseitige Tabellen
% \usepackage[svnon,svnfoot]{svnver} % SVN Versionsinformation 
%% ---------------- end of usepackages -------------

%\svnversion{$Id: thesis.tex 65 2012-05-10 10:32:11Z bless $} % In case that you want to include version information in the footer

%% Informationen für die PDF-Datei
\hypersetup{
 pdfauthor={Sven Fiergolla},
 pdftitle={Improvements on RLE by preprocessing}
 pdfsubject={Not set},
 pdfkeywords={Not set}
}

% Macros, nicht unbedingt notwendig
\input{macros}

% Print URLs not in Typewriter Font
\def\UrlFont{\rm}

\newcommand{\blankpage}{% Leerseite ohne Seitennummer, nächste Seite rechts
 \clearpage{\pagestyle{empty}\cleardoublepage}
}
\newcommand{\DFA}{\textsc{DFA}\xspace}
\newcommand{\DFAs}{\textsc{DFAs}\xspace}
\newcommand{\ExpSpace}{ExpSpace\xspace}
\newcommand{\LogSpace}{LogSpace\xspace}



%% Einstellungen für das gesamte Dokument

% Trennhilfen
% Wichtig! 
% Im ngerman-paket sind zusätzlich folgende Trennhinweise enthalten:
% "- = zusätzliche Trennstelle
% "| = Vermeidung von Ligaturen und mögliche Trennung (bsp: Schaf"|fell)
% "~ = Bindestrich an dem keine Trennung erlaubt ist (bsp: bergauf und "~ab)
% "= = Bindestrich bei dem Worte vor und dahinter getrennt werden dürfen
% "" = Trennstelle ohne Erzeugung eines Trennstrichs (bsp: und/""oder)

% Trennhinweise fuer Woerter hier beschreiben
\hyphenation{
% Pro-to-koll-in-stan-zen
}

% Index-Datei öffnen
\ifnotdraft{\makeindex}

\begin{document}

\frontmatter
\pagenumbering{roman}
\ifnotdraft{
 %% Titelseite
%% Vorlage $Id: titelseite.tex 61 2012-05-03 13:58:03Z bless $

\def\usesf{}
\let\usesf\sffamily % diese Zeile auskommentieren für normalen TeX Font

\newsavebox{\Erstgutachter}
\savebox{\Erstgutachter}{\usesf Prof.~Dr.~Henning Fernau}
\newsavebox{\Zweitgutachter}
\savebox{\Zweitgutachter}{\usesf  Dr.~Petra Wolf}
\newsavebox{\Betreuer}
\savebox{\Betreuer}{\usesf Prof.~Dr.~Henning Fernau \& Dr.~Petra Wolf}

\begin{titlepage}
\setlength{\unitlength}{1pt}
\begin{picture}(0,0)(85,770)
%\includegraphics[width=\paperwidth]{logos/KIT_Deckblatt}
\end{picture}

\thispagestyle{empty}

%\begin{titlepage}
%%\let\footnotesize\small \let\footnoterule\relax
\begin{center}
\hbox{}
\vfill
{\usesf
{\huge\bfseries Theoretical and Empirical Analysis of the Representability of Dynamic Networks as Edge Periodic Temporal Graphs \par}
\vskip 1.8cm
\begin{otherlanguage}{ngerman}
{\huge Masterarbeit}\\
\vskip 0.5cm
zur Erlangung des akademischen Grades\\
Master of Science (M.Sc.) 
\vskip 1.5cm

{\large Universität Trier\\
FB IV - Informatikwissenschaften\\
Lehrstuhl für Theoretische Informatik\\}

\vskip 3cm
\begin{tabular}{p{3.5cm}l}
Gutachter: & \usebox{\Erstgutachter} \\
 & \usebox{\Zweitgutachter} \\
Betreuer: & \usebox{\Betreuer} \\
\end{tabular}
\vskip 3cm
Vorgelegt am \today~ von:\\
\vskip .5cm
Sven Fiergolla\\
Kopernikusstr 23\\
10245 Berlin\\
sven.fiergolla@gmail.com\\
Matr.-Nr. 1252732
\end{otherlanguage}
}
\end{center}
\vfill
\end{titlepage}
%% Titelseite Ende


%%% Local Variables: 
%%% mode: latex
%%% TeX-master: "thesis"
%%% End: 

 \blankpage % Leerseite auf Titelrückseite
 \chapter*{Abstract}
%% ==============================
Edge periodic temporal graphs (EPGs) are graphs edge availability varies over time, that can represent the periodic behavior of dynamic networks.
In EPGs, for each edge $e$ of the graph, a binary string $s_e$ determines in which time steps the edge is present, namely, $e$ is present in time step $t$ if and only if $s_e$ contains a 1 at position $t~ mod~ |s_e|$.
Due to their periodic nature, EPGs can be a more concise way of representing complex periodic systems compared to traditional temporal graphs.
In this paper, we study the effectiveness of representing dynamic networks as EPGs theoretically and empirically, by interpreting the labels as unary automata and applying a decomposition algorithm for DFAs as well as well as two novel algorithms that identify other, more \enquote{explainable} factors for human interpreters.

\paragraph{Übersetzung}
Edge-periodische Zeitgraphen (EPGs)  sind Graphen, deren Kantenverfügbarkeit sich im Laufe der Zeit ändert und die das periodische Verhalten dynamischer Netzwerke darstellen können.
In EPGs bestimmt für jede Kante $e$ des Graphen eine binäre Zeichenkette $s_e$, in welchen Zeitschritten die Kante vorhanden ist; das heißt, $e$ ist im Zeitschritt $t$ vorhanden, wenn und nur wenn $s_e$ an Position $t~ mod~ |s_e|$ eine 1 enthält.
Aufgrund ihrer periodischen Natur können EPGs eine prägnantere Möglichkeit darstellen, komplexe periodische Systeme im Vergleich zu herkömmlichen zeitlichen Graphen zu repräsentieren. In dieser Arbeit untersuchen wir theoretisch und empirisch die Effektivität der Darstellung dynamischer Netzwerke als EPGs, indem wir die Labels als unäre Automaten interpretieren und einen Dekompositionsalgorithmus für DFAs anwenden, sowie zwei neuartige Algorithmen, die andere, für menschliche Interpreten potenziell \enquote{erklärbarere} Faktoren identifizieren.
}
%
%% *************** Hier geht's ab ****************
%% ++++++++++++++++++++++++++++++++++++++++++
%% Verzeichnisse
%% ++++++++++++++++++++++++++++++++++++++++++
\ifnotdraft{
{\parskip 0pt\tableofcontents} % toc bitte einzeilig
%\blankpage

\listoffigures
%\blankpage
\listoftables
%\blankpage
}


%% ++++++++++++++++++++++++++++++++++++++++++
%% Hauptteil
%% ++++++++++++++++++++++++++++++++++++++++++
\graphicspath{{img/}}

\mainmatter
\pagenumbering{arabic}
\chapter{Introduction}
\label{ch:Introduction}
%% ==============================
%% ==============================
\section{Motivation}
%% ==============================
\label{ch:Introduction:sec:Motivation}

%% ==============================
\section{Problem statement}
%% ==============================
\label{ch:Introduction:sec:Problem statement}


%% ==============================
\section{Structure of this Work}
%% ==============================
\label{ch:Intoduction:sec:Structure}
\par{
This work is structured into a first introduction into the basics of EPGs and an analysis of the current state of the art. Then, the conceptual design is depicted with a following analysis of the results. Afterwards, the implementation of the algorithms are described and the work as a whole is evaluated with a short closing discussion.  
}
  % Einleitung
\chapter{Preliminaries}
\label{ch:Preliminaries}
%% ==============================

\section{Notation}
For a string $w = w_0w_1 \dots w_n$ with $w_i \in \{0, 1\}$, for $0 \leq i \leq n$, $w[i]$ represents the symbol $w_i$ at position $i$ in $w$. Let the length of $w$ be $|w| = n$. Concatenation of strings $u$ and $v$ is written as $u \cdot v$. An edge periodic temporal graph, $G = (V, E, \tau)$ (see also \cite{erlebach2020game}) consists of a graph $G = (V, E)$ called the underlying graph and a function $\tau : E \rightarrow \{0, 1\}$ where $\tau$ maps each edge $e$ to a string $\tau(e) \in \{0, 1\}$, such that $e$ exists in a time step $t \geq 0$ if and only if $\tau(e)[t] = 1$, where $\tau(e)[t] := \tau(e)[t~ mod~ |\tau(e)|]$. Every edge $e$ exists in at least one time step, that is, for each edge $e$ there is some $t_e \in [0, |\tau(e)| - 1]$ with $\tau(e)[t_e] = 1$. We abbreviate $i$ repetitions of identical symbols $\sigma$ in $\tau(e)$ as $\sigma^i$. We denote the subgraph of $G$ in time step $t$ with $G(t)$.


\section{State of the art}

The study of temporal graphs, also known as time-varying or dynamic graphs, has gained significant attention in recent years. Researchers have been actively exploring various aspects of temporal graphs, including their representation \cite{Holme_2012}, analysis \cite{DBLP:journals/corr/Erlebach0K15}\cite{DBLP:journals/corr/Michail15}, and mining techniques. Temporal graph mining focuses on extracting valuable information from temporal graphs. Researchers have developed techniques for tasks such as community detection, which aims to identify groups of nodes that exhibit cohesive patterns of interactions over time (e.g. graph minors or subgraphs) \cite{arrighi2022multiparameter}. Link prediction techniques have also been explored to predict future edge occurrences based on historical data \cite{temporalLinkPrediction}. Additionally, various metrics have been proposed to quantify important properties of temporal graphs, such as temporal centrality and clustering coefficients \cite{temporalClusterCoefficient}. Understanding the evolution and dynamics of temporal graphs is another research area. Models have been developed to simulate the evolution of temporal graphs, considering factors such as edge arrival, departure, and temporal dependencies. The study of influence and cascading behavior in temporal graphs explores how information or influence spreads over time, including modeling cascades and studying diffusion processes \cite{temporalEvolution}. Temporal network analysis involves the development of measures and techniques to characterize and visualize temporal graphs. Various metrics have been proposed to capture properties like assortativity, modularity, and temporal path analysis. Visualization techniques have been designed to effectively represent and explore the dynamic nature of temporal graphs \cite{kerracher2014design}. Temporal graph research finds applications in diverse fields. For example, it has been applied to study social networks, analyzing the evolution of friendships, communities, and information diffusion. In transportation systems, temporal graphs help model traffic patterns, optimize routes, and analyze network dynamics \cite{tang2009temporal}. In the realm of biology, temporal graphs have been used to study protein-protein interaction networks \cite{fu2022dppin}, gene regulatory networks, and other biological systems to understand dynamic processes \cite{dibrita2022temporal}.

Overall, the study of temporal graphs is a multidisciplinary field, with researchers from computer science, network science, physics, and other domains contributing to its advancements. Ongoing research continues to explore new techniques and methodologies for analyzing, modeling, and understanding the complex dynamics of temporal graphs.  % Grundlagen
\chapter{Decompositions of Edge Labels by decomposing Unary DFAs}
\label{ch:Analysis}
%% ==============================

As our goal is to represent general dynamic networks and temporal graphs as EPGs, one problem is the missing periodicity in general temporal and dynamic graphs. 

\begin{figure}[h]
	\begin{minipage}[t]{0.49\textwidth}
		\centering
		\begin{tikzpicture}[every edge quotes/.style = {auto, font=\footnotesize, sloped}]
			\begin{scope}[every node/.style={circle,fill}]
				\node (A) at (0,0) {};
				\node (B) at (0,2) {};
				\node (C) at (2,0) {};
				\node (D) at (2,2) {};
				\node (E) at (1,3) {};
			\end{scope}
			
			\draw (A) edge["101\dots"] (B)
			(B) edge["100\dots"] (C)
			(D) edge["001\dots"] (C)
			(A) edge["000\dots"] (C)
			(D) edge["111\dots"] (E)
			(B) edge["011\dots"] (D)
			(B) edge["101\dots"] (E);
		\end{tikzpicture}\\
	Temporal graph with long labels
	\end{minipage}
	\begin{minipage}[t]{0.49\textwidth}
		\centering
		\begin{tikzpicture}[every edge quotes/.style = {auto, font=\footnotesize, sloped}]
			\begin{scope}[every node/.style={circle,fill}]
				\node (A) at (0,0) {};
				\node (B) at (0,2) {};
				\node (C) at (2,0) {};
				\node (D) at (2,2) {};
				\node (E) at (1,3) {};
			\end{scope}
			
			\draw (A) edge[bend left, below,"101"] (B)
			(A) edge[bend right, below,"0001"] (B)
			(B) edge["100"] (C)
			(D) edge[bend right, below,"0100"] (C)
			(D) edge[bend left, below,"10"] (C)
			(A) edge["0"] (C)
			(D) edge[bend right, above,"110"] (E)
			(B) edge[bend right, below,"01111"] (D)
			(B) edge["10"] (E)
			(E) edge[bend right, below,"01"]  (D);
		\end{tikzpicture}\\
	EPG with multi edges but short labels
	\end{minipage}
\caption{Desired transformation of temporal graphs into EPGs}
\end{figure}

To transform a temporal graph, a shortest possible string $w'$ which is equal to the original label at every time step $\forall t \geq 0, \tau(e)[t] = w'[t]$ has to be found. This $w'$ can itself be composed by combining different factors $w_1,w_2,\dots,w_n$. Initially, we start with a temporal graph where all edge labels $\tau(e)$ have fixed length and there are no periods present. To find and analyze such periods in the given labels, the algorithm from \cite{DBLP:journals/corr/abs-2107-04683} is used and adapted to our problem. To apply the algorithm which is defined for automata, a label $ w \in \{0,1\}^*$ is interpreted as an unary automata. In the label either the $0s$ or the $1s$ symbols are used to represent final states $Q_f$. The algorithm from \cite{DBLP:journals/corr/abs-2107-04683} can be simplified due to the fact that we only represent unary automata with \textbf{$|\Sigma| = 1$} and therefore only have a single transition from each state, basically forming a simple circle of all possible states. Using the fact that our alphabet is of size one, we only need to follow a single transition and furthermore, we only need to check multiples of the chosen period. This means that for a period length of $i$ we only have to check $i$ states on the circle being in the same state.

\section{Decompositions of DFAs}
\label{sec:decomposition-unary-dfas}

\begin{figure}[h]
	\begin{minipage}[t]{0.29\textwidth}
		\centering
		Label $w=1001$
	\end{minipage}
	\begin{minipage}[t]{0.69\textwidth}
		\centering
		\begin{tikzpicture}[shorten >=1pt,node distance=2cm,on grid,auto] 
			\node[state,initial,accepting] (q_0)   {$q_0$}; 
			\node[state] (q_1) [right=of q_0] {$q_1$}; 
			\node[state] (q_2) [right=of q_1] {$q_2$}; 
			\node[state,accepting](q_3) [right=of q_2] {$q_3$};
			\path[->] 
			(q_0) edge  node {} (q_1)
			(q_1) edge  node {} (q_2)
			(q_2) edge  node {} (q_3)
			(q_3) edge[bend right, above]  node {} (q_0);
		\end{tikzpicture}
		\begin{tikzpicture}[shorten >=1pt,node distance=2cm,on grid,auto] 
			\node[state,initial] (q_0)   {$q_0$}; 
			\node[state,accepting] (q_1) [right=of q_0] {$q_1$}; 
			\node[state,accepting] (q_2) [right=of q_1] {$q_2$}; 
			\node[state](q_3) [right=of q_2] {$q_3$};
			\path[->] 
			(q_0) edge  node {} (q_1)
			(q_1) edge  node {} (q_2)
			(q_2) edge  node {} (q_3)
			(q_3) edge[bend right, above]  node {} (q_0);
		\end{tikzpicture}
	\end{minipage}
	\caption{Equivalence of binary strings and unary permutation \DFAs}
\end{figure}

A DFA $\mathcal{A}$ is composite if its language $L(\mathcal{A})$ can be decomposed into an intersection $\cup^k_{i=1} L(\mathcal{A}_i)$ of languages of smaller DFAs. Otherwise, $\mathcal{A}$ is prime. This notion of \textit{primality} was introduced by Kupferman and Mosheiff in \cite{prime-languages}, and they proved that we can decide whether a DFA is composite in ExpSpace and later in \cite{unara-prime-languages}, the decomposition question for unary DFAs was proven to be in \LogSpace. In the paper~\cite{DBLP:journals/corr/abs-2107-04683} by Jecker, Mazzocchi and Wolf, they provided a \LogSpace algorithm for commutative permutation DFAs, if the alphabet size is fixed, which also puts the bounded $k$-composite question for unary DFAs in the \LogSpace complexity class, see \ref{tab:dfa-decomp-complexity} for further reference. The decomposition of unary \DFAs are usually characterized by means of clean quotients. Let  $A = \lbrace\Sigma, Q, q_I , \sigma, F\rbrace$ be a unary-DFA. A clean quotient $A_d$ of $A$ is a DFA obtained by folding its cycle of length $l$ to a cycle of length $d$, for some strict divisor $d$ of $l$. Formally, $A_d$ is induced by the equivalence relation $\sim_d$ defined by

\[ q_i \sim_d q_j ~\text{if and only if}~i \equiv j ~mod~ d \]

Since $\sim_d$ is coherent with $\sigma$, and therefore $L(A) \subseteq L(A_d)$.

\begin{figure}[h]
	\begin{minipage}[t]{\textwidth}
		\centering
		\begin{tikzpicture}[shorten >=1pt,node distance=2cm,on grid,auto] 
			\node[state,initial] (q_0)   {$q_0$}; 
			\node[state,accepting] (q_1) [ right=of q_0] {$q_1$}; 
			\node[state] (q_2) [ right=of q_1] {$q_2$}; 
			\node[state,accepting] (q_3) [ right=of q_2] {$q_3$};
			\node[state](q_4) [ right=of q_3] {$q_4$};
			\node[state](q_5) [ right=of q_4] {$q_5$};
			\path[->] 
			(q_0) edge  node {} (q_1)
			(q_1) edge  node {} (q_2)
			(q_2) edge  node {} (q_3)
			(q_3) edge  node {} (q_4)
			(q_4) edge  node {} (q_5)
			(q_5) edge[bend right, above]  node {} (q_0);
		\end{tikzpicture}		
	\end{minipage}
	\begin{minipage}[b]{0.49\textwidth}
		\centering
		\begin{tikzpicture}[shorten >=1pt,node distance=2cm,on grid,auto] 
		\node[state,initial,accepting] (q_0)   {$q_0$}; 
		\node[state,accepting] (q_1) [right=of q_0] {$q_1$}; 
		\node[state](q_2) [right=of q_1] {$q_2$};
		\path[->] 
		(q_0) edge  node {} (q_1)
		(q_1) edge  node {} (q_2)
		(q_2) edge[bend right, above]  node {} (q_0);

\end{tikzpicture}
	\end{minipage}
	\begin{minipage}[b]{0.49\textwidth}
		\centering
		\begin{tikzpicture}[shorten >=1pt,node distance=2cm,on grid,auto] 
		\node[state,initial] (q_0)   {$q_0$}; 
		\node[state,accepting] (q_1) [right=of q_0] {$q_1$}; 
		\path[->] 
		(q_0) edge  node {} (q_1)
		(q_1) edge[bend right, above]  node {} (q_0);
		\end{tikzpicture}
	\end{minipage}
	\caption{The DFA $A$ and its clean quotients $A_2$ and $A_3$}
	\label{fig:clean-quotients}
\end{figure}

In the example in figure \ref{fig:clean-quotients}, the \DFA $A$ with final states $q_1$ and $q_3$ can be replaced by the clean quotients $A_2$ and $A_3$ as there is no $q_i \in F$ in $A$ where both clean quotients $A_d$ are not in a final state $q_{i ~mod~ d}$. Note that the coherency of $\sim$ with respect to $\sigma$ guarantees that the definition of $\sigma'$ is independent of the choice of the state $p$ in $[p]$. On the other hand, we do not require states related by $\sim$ to agree on membership in $F$, and define $F_d$ so that the language of $A_d$ over-approximates that of $A$. Formally, $L(A) \subseteq L(A_d)$, as every accepting run of $A$ induces an accepting run of $A_d$. In this example this is visible and there is a state $q_0$ in $A$ where the clean quotient $A_2$ is in a final state but $A_3$ is not and only the combination of both clean quotients is required to fully decompose the original \DFA. Finding clean quotients is trivial but the number of potential quotients rises linear with the number of states or the length of the edge label.

\subsection{Decomposing using maximal divisors}
TODO: explain factors vs quotients
\begin{figure}[h]
	\begin{minipage}[t]{\textwidth}
		\centering
		\begin{tikzpicture}[shorten >=1pt,node distance=2cm,on grid,auto] 
			\node[state,initial] (q_0)   {$q_0$}; 
			\node[state,accepting] (q_1) [ right=of q_0] {$q_1$}; 
			\node[state] (q_2) [ right=of q_1] {$q_2$}; 
			\node[state,accepting] (q_3) [ right=of q_2] {$q_3$};
			\node[state,accepting](q_4) [ right=of q_3] {$q_4$};
			\node[state,accepting] (q_5) [ right=of q_4] {$q_5$};
			\path[->] 
			(q_0) edge  node {} (q_1)
			(q_1) edge  node {} (q_2)
			(q_2) edge  node {} (q_3)
			(q_3) edge  node {} (q_4)
			(q_4) edge  node {} (q_5)
			(q_5) edge[bend right, above]  node {} (q_0);
		\end{tikzpicture}		
	\end{minipage}
	\begin{minipage}[b]{0.49\textwidth}
		\centering
		\begin{tikzpicture}[shorten >=1pt,node distance=2cm,on grid,auto] 
			\node[state,initial] (q_0)   {$q_0$}; 
			\node[state,accepting] (q_1) [right=of q_0] {$q_1$}; 
			\node[state](q_2) [right=of q_1] {$q_2$};
			\path[->] 
			(q_0) edge  node {} (q_1)
			(q_1) edge  node {} (q_2)
			(q_2) edge[bend right, above]  node {} (q_0);
			
		\end{tikzpicture}
		\end{minipage}
		\begin{minipage}[b]{0.49\textwidth}
		\centering
		\begin{tikzpicture}[shorten >=1pt,node distance=2cm,on grid,auto] 
			\node[state,initial] (q_0)   {$q_0$}; 
			\node[state,accepting] (q_1) [right=of q_0] {$q_1$}; 
			\path[->] 
			(q_0) edge  node {} (q_1)
			(q_1) edge[bend right, above]  node {} (q_0);
			
		\end{tikzpicture}
	\end{minipage}
	\caption{The DFA $A$ and its clean factors $A_2$ \& $A_3$}
	\label{fig:clean-quotients}
\end{figure}


Usually the algorithms and proofs are considering unary \DFAs consisting of a chain leading into a cycle of states. Since we obtain our \DFAs by transforming from a periodic label, we only have \DFAs with empty chains, therefore only considering unary permutation \DFA. This allows us to use a slightly simplified version of the Algorithm from \cite{DBLP:journals/corr/abs-2107-04683} as seen in Algorithm \ref{algo:composite} as we do not encounter unary automata with $\sigma(q, uv) \not = \sigma(q, vu)$. Additional we are not only interested in answering the yes/no question of the composite problem but we actually want to collect the factors and continue our computation.

\begin{algorithm}[H]
	\label{algo:composite}
	\DontPrintSemicolon
	\SetKwProg{Fn}{Function}{:}{}
	\Fn{getComposite($A = ⟨{a}, Q, qI , \sigma, F ⟩ $: unary DFA, integer k)}{
		FactorList $\gets \emptyset$\; 
		\ForEach{binaryString $\in \{0,1\}^{log|Q|}$ with $\leq k$ ones}{
			\If{isFactor($A$,binaryString)}{FactorList.add(binaryString)}
		}
		\KwRet FactorList\;
	}

	\Fn{isFactor($A = ⟨{a}, Q, qI , \sigma, F ⟩ $: unary DFA, binaryString)}{
		\ForEach{$q \in Q \setminus F$}{
			\If{not cover(A,q,binaryString)}{\KwRet false}
			\KwRet true
		}
	}
	
	\Fn{cover($A = ⟨{a}, Q, qI , \sigma, F ⟩ $: unary DFA, binaryString, $q \in Q \setminus F$)}{
		\ForEach{$i$ with wordCombination[i]=1}{
			$p_1 \gets i$'th prime divisor of $|Q|$\;
			\If{cover($A,q,\sigma(q,a^{|Q|/p_i})$)}{\KwRet true}
		}
		\KwRet false
	}
	
	\caption{LOGSPACE-algorithm solving the Decomp problem for unary DFAs and returning the factors.}
\end{algorithm}

TODO: explain algo and difference to original

Regarding the complexity of the given algorithm, since given 2 divisors $i^1$ and $i^2$ of $|Q|$, with $i^1 < |Q|$ divides $i^2 < |Q|$, then all states covered by $a^{i1}$ are also covered by $a^{i2}$. Therefore we now only consider words of the form $a^i$ where $i$ is a maximal divisor of $|Q|$ as potential candidates for the decomposition. Now, let $p_1^{j1} \cdot p_2^{j2} \cdot ~ \dots ~\cdot p_m^{jm} = |Q|$ be the prime factor decomposition of $|Q|$. Since $|Q|$ is given in unary we can compute the prime factor decomposition of $|Q|$ in space logarithmic in $|Q|$. $A$ is $k$-factor composite if and only if a selection of $k$ words from the set $W = \{a^{|Q|/p_i} | 1 \leq i \leq m\}$ cover all the rejecting states of $A$.  As $|W| = m$ is logarithmic in $|Q|$, we can iterate over all sets in $2^W$ of size at most $k$ in \LogSpace. By Lemma 6 from \cite{DBLP:journals/corr/abs-2107-04683}, checking whether a state $q \in Q$ is covered by the current collection of $k$ words can also be done in \LogSpace. The original \LogSpace-algorithm is described in \cite{DBLP:journals/corr/abs-2107-04683} in Algorithm 3.

TODO: outlier handling

\section{Explainability metrics \& measurement}

\subsection{Short Factors}

\subsection{Fourier Transform}


\section{Modified Decomposition Problem}
\begin{figure}[h]
	\includegraphics[width=\linewidth]{proof-sketches/Screenshot[2]-01.png}
\end{figure}

     % Analyse
%% entwurf.tex
%% $Id: entwurf.tex 61 2012-05-03 13:58:03Z bless $
%%

\chapter{Decomposition Algorithm}
\label{ch:Conceptual Design}
%% ==============================

%%% Local Variables: 
%%% mode: latex
%%% TeX-master: "thesis"
%%% End: 
     % Entwurf
%\include{src/implemen}    % Implementierung
%% eval.tex
%% $Id: eval.tex 61 2012-05-03 13:58:03Z bless $

\chapter{Evaluation}
\label{ch:Evaluation}
%% ==============================

\begin{figure}[h]
	\includegraphics[width=\linewidth]{../old_plots/backup/SHORTEST_PERIODS-all-relative-values-by-factor-size.png}
\end{figure}
\begin{figure}
	\includegraphics[width=\linewidth]{../plots/GREEDY_SHORT_FACTORS-all-values-by-factor-size.png}
\end{figure}


%%% Local Variables: 
%%% mode: latex
%%% TeX-master: "thesis"
%%% End: 
        % Evaluation
%% zusammenf.tex
%% $Id: zusammenf.tex 61 2012-05-03 13:58:03Z bless $
%%

\chapter{Discussion}
\label{ch:Discussion}
%% ==============================
TODO:\\
- evaluate and vs or and data set\\
- contextualize evaluation\\
- outlook on explainability analysis\\

%%% Local Variables: 
%%% mode: latex
%%% TeX-master: "thesis"
%%% End: 
   	  % Diskussion und Ausblick

%% ++++++++++++++++++++++++++++++++++++++++++
%% Anhang
%% ++++++++++++++++++++++++++++++++++++++++++

\appendix
%\include{anhang_a}
%\include{anhang_b}

%% ++++++++++++++++++++++++++++++++++++++++++
%% Literatur
%% ++++++++++++++++++++++++++++++++++++++++++
%  mit dem Befehl \nocite werden auch nicht 
%  zitierte Referenzen abgedruckt

\cleardoublepage
\phantomsection
\addcontentsline{toc}{chapter}{\bibname}
%%
%%\nocite{*} % nur angeben, wenn auch nicht im Text zitierte Quellen 
           % erscheinen sollen

% spezielle Zitierstile: Labels mit vier Buchstaben und Jahreszahl
%\bibliographystyle{itmalpha}  % ausgeschriebene Vornamen der Autoren
%\bibliographystyle{cpc} % Use for final print and local pdf gen, does not work on github cli
\bibliographystyle{itmabbrv} % Use for cli - mit abgekürzten Vornamen der Autoren

\bibliography{thesis}

%% ++++++++++++++++++++++++++++++++++++++++++
%% Index
%% ++++++++++++++++++++++++++++++++++++++++++
\ifnotdraft{
\cleardoublepage
\phantomsection
\printindex            % Index, Stichwortverzeichnis
}

 %
 % Die folgende Erklärung ist für Diplomarbeiten Pflicht
 % (siehe Prüfungsordnung), für Studienarbeiten nicht notwendig
 \include{src/erklaerung}
 \blankpage % Leerseite auf Erklärungsrückseite
 
\end{document}
%% end of file
